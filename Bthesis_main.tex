\documentclass[a4j,12pt]{jreport}
%\documentclass{jreport}
\usepackage[dvipdfmx]{graphicx}
\usepackage{amsmath,amssymb}
\usepackage{here}
\usepackage{algorithm}
\usepackage{algpseudocode}
\usepackage{hhline}
\usepackage{makecell}
\usepackage{tabularx}
\usepackage[hang,small,bf]{caption}
\usepackage[subrefformat=parens]{subcaption}
\usepackage{url}
\captionsetup{compatibility=false}

\def\syaji{ \chapter*{謝辞} \addcontentsline{toc}{chapter}{謝辞}}
\renewcommand{\bibname}{参考文献}
\setlength{\textheight}{\paperheight}
\setlength{\topmargin}{4.6mm}
\addtolength{\topmargin}{-\headheight}
\addtolength{\topmargin}{-\headsep}
\addtolength{\topmargin}{-\headheight}
\addtolength{\textheight}{-60mm}

\setlength{\textwidth}{\paperwidth}
\setlength{\oddsidemargin}{-0.4mm}
\setlength{\evensidemargin}{-0.4mm}
\addtolength{\textwidth}{-50mm}

\begin{document}

%%%%%%%%%%%%%%%%%%%%%
% 表紙
%%%%%%%%%%%%%%%%%%%%%
\thispagestyle{empty}
\begin{center}
  \begin{Large}
    \vspace*{0.7cm}
    {\large 卒業研究論文}\\
    \vspace*{2.5cm}
    {\LARGE\bf JR最安分割乗車券探索システムの開発}\\
    \vspace*{7.5cm}
    東條 涼平\\
    学籍番号\hspace*{1zw}22D8102015D\\
    \vspace*{2.5cm}
    中央大学理工学部情報工学科\hspace*{1zw} アルゴリズム理論基礎研究室\\
    \vspace*{3.0cm}
    2026年3月\\
  \end{Large}
\end{center}


%%%%%%%%%%%%%%%%%%%%%
% 概要
%%%%%%%%%%%%%%%%%%%%%
\newpage
\renewcommand{\baselinestretch}{1.25} \selectfont
\pagenumbering{roman}


\begin{center} {\large \bf{概 要}} \end{center}

JR線を利用する際,安く移動をしようとするとき,通しで購入する乗車券の運賃よりも乗車区間を途中で区切って複数枚の乗車券を購入するほうが安い場合がある.
この区間を分けて購入した乗車券を分割乗車券という.
従来,任意の駅間において分割乗車券を考慮したときに最安解を出すためには,最適解のルートとして考えられる経路の全てで分割乗車券の運賃を比較する必要があったが,このとき分割乗車券の経路が最短経路でない場合もあるため,複数の経路を試す必要があった.
そこで,Dijkstra's algorithm\cite{dijkstra}とYen's Algorithm\cite{yen}を使って最安となりうる経路の候補を出し,それぞれに対して分割経路を計算をして,それらの結果をもとに最安解を出力することにより,経路を指定せずに発駅と着駅の入力のみで最適解を出すプログラムを作成した.

\vspace{1zw} \noindent
{\bf キーワード: }分割乗車券,経路探索,JR運賃計算,グラフ理論

%%%%%%%%%%%%%%%%%%%%%
% 目次
%%%%%%%%%%%%%%%%%%%%%
\tableofcontents


\newpage
\pagenumbering{arabic}

%%%%%%%%%%%%%%%%%%%%%
% 1章
%%%%%%%%%%%%%%%%%%%%%
\chapter{序論} \label{chapter:intro}

\section{研究の背景}
JRでは営業キロを一定の幅に区切って運賃を設定しているため,乗車券の区間を分割して購入したほうが安くなる場合がある.
区間を分けて乗車券を購入することを分割乗車券といい,2000年以前からインターネット上で知られていた.
発駅と着駅のほかに経路を指定した場合であれば,最安解を出力する分割乗車券プログラム\cite{oba}が既に存在する.
多くの場合では,発駅から着駅までを結ぶ最短経路上に存在するが,都心のような路線が狭い範囲で路線が密集している場合など,最短経路上に分割駅が無い場合もある.
そのため,旅客が最安解である経路を特定することは困難である.

\section{研究の対象}
今回の研究では,大人の片道乗車券を対象とする.
また,新幹線を使った経路では乗車券の他に特急券が必要となるため今回の研究では在来線のみを探索の対象とする.
BRT線においても,発売に制限があるため探索の対象外とする.
片道乗車券の運賃計算には多くの例外があり複雑である.
例えば,同一の区間であっても複数の正規運賃が存在することもある.
この研究で扱う乗車券の運賃と経路は,運賃計算の手順が記載されている旅客営業規則に則って計算をして,駅の窓口や自動券売機で実際に購入できるものを扱い,ICカード運賃は考慮しない.

\section{関連研究・従来の手法}
分割乗車券プログラムをWebアプリとして公開しているサイトが存在する.このサイトでは,発駅と着駅の他に経路を指定する必要があり,その経路上で最安分割乗車券を探索する.
また,鉄道運賃の計算に関する研究として,森田ら\cite{morita}は,運賃計算ルールをグラフ構造に反映させたFarenetを提案し,高速な最安経路探索を実現している.
さらに,分割乗車券に特化した先行研究として,平田ら\cite{hirata}によるスマートフォンに特化した乗車券分割アプリの事例がある.
しかし,平田らの手法では最短経路上で分割回数が1回に限定されているという制約があり,探索エリアも関西の一部エリアのみである.

\section{研究の目的と意義}
旅客らに実用的な節約手段を提供でき,複雑な制約を持つ大規模グラフ問題に対する効率的な解法を高速な計算時間で実装する.

%%%%%%%%%%%%%%%%%%%%%
% 2章
%%%%%%%%%%%%%%%%%%%%%
\chapter{JR運賃制度と問題の定義} \label{chapter:fare_system}

\section{JR運賃計算の基礎}
JRは6つの旅客鉄道会社などから構成される.
この研究では旅客鉄道会社のみを扱い,これ以降は通称を使う.
表\ref{tab:companies}に各社の通称を示す.

\begin{table}[H]
\centering
\caption{旅客鉄道会社一覧}
\label{tab:companies}
\begin{tabular}{|c|c|} \hline
会社名 & 通称 \\ \hline
北海道旅客鉄道 & JR北海道 \\ \hline
東日本旅客鉄道 & JR東日本 \\ \hline
東海旅客鉄道 & JR東海 \\ \hline
西日本旅客鉄道 & JR西日本 \\ \hline
四国旅客鉄道 & JR四国 \\ \hline
九州旅客鉄道 & JR九州 \\ \hline
\end{tabular}
\end{table}

また,JR東日本・JR東海・JR西日本をまとめて本州三社と呼ぶ.各旅客鉄道会社に旅客営業規則が定められているが,それぞれに内容の差異は無い.
この研究では,JR東日本の旅客営業規則を参照する.\\

まずは,片道乗車券の定義について解説をする.
片道乗車券における経路は重複してはならず,環状線を超える経路であってはならない.
ここで,環状線を超えるとは,発駅から一駅ずつ経路を進めて行った場合に閉路ができたらその駅までの乗車券は片道乗車券として発売可能であり,それより先に経路を伸ばした場合は発売ができないことを示す.
つまり,発着駅が同じとなる円状の経路は発売可能である.

次に,運賃計算をする手順を以下に示す.

JRの路線は「幹線」と「地方交通線」のいずれかに分類され,各駅間には営業キロが設定されている.
幹線のみや地方交通線のみを乗車する場合は営業キロを使い,それぞれの賃率により運賃計算をする.
幹線と地方交通線を連続して乗車する場合は,運賃を計算する上で幹線と地方交通線の間で賃率が違うことから,地方交通線の乗車区間に対する営業キロに賃率比に応じて換算したものを,JR北海道,JR東日本,JR東海,JR西日本では「賃率換算キロ」といい,JR四国,JR九州では「擬制キロ」という.
これらは,地方交通線の各駅間に設定されており,幹線の乗車区間に対する営業キロと地方交通線の乗車区間に対する「賃率換算キロ」及び「擬制キロ」を合算したものを「運賃計算キロ」という.
また,一部区間を相互発着となる区間に乗車する場合は,運賃の他に鉄道駅バリアフリー料金が収受されるが,今回の研究では,運賃に鉄道駅バリアフリー料金をあわせた金額を運賃として扱う.

\section{運賃経路を外方へ伸ばしたほうが運賃が安い場合}
運賃は通常,発駅から着駅までの経路区間に対する運賃計算キロで計算される.
よって,経路を外方へ伸ばしたときキロ数も増え,安くなることは無い.
しかし,以下に示す特例が適用されるときは運賃計算経路が変わるため,運賃経路を外方へ伸ばしたほうが運賃が安い場合が存在する.

\section{問題の定式化}
発駅と着駅のみを入力として,その間を最も安く移動するような分割乗車券を出力する.このとき,乗車券の最大分割数は無限として,最安分割乗車券の候補となる経路は全て探索する.

%%%%%%%%%%%%%%%%%%%%%
% 3章
%%%%%%%%%%%%%%%%%%%%%
\chapter{探索手法} \label{chapter:algorithm}

\section{運賃計算プログラム}
分割乗車券プログラムから呼び出されるプログラムであり,入力では発駅と着駅の他に経路も指定する必要がある.

運賃計算アルゴリズムは以下の通りである.
\begin{itemize}
    \item 特定運賃区間と経路が一致する場合はその運賃を返す.
    \item 入力された経路が,山手線内の駅相互発着の場合,東京附近における電車特定区間内相互発着の場合,大阪附近における電車特定区間内相互発着の場合はそれぞれに対応した賃率により運賃を計算する.
    \item 全ての駅間を旅客会社ごとに分けて,まずはすべての区間を本州三社の運賃体系で計算をする.
    JR北海道,JR四国,JR九州のそれぞれの区間で,会社ごとの運賃から本州三社の運賃の差を加算額として加える.
    また,加算運賃区間にまたがる場合は加算運賃を加えた額を返す.
\end{itemize}

運賃計算をした上で,鉄道バリアフリー料金該当区間相互発着の場合は鉄道バリアフリー料金を加えた額を運賃とする.\\

この運賃計算プログラムでは,旅客営業規則の通りに,入力された経路が計算される.
しかし,その経路の始発駅や終着駅を外方の駅として運賃経路を伸ばしたときに運賃が安くなる場合は,伸ばした駅を乗車券の始発駅や終着駅として安い運賃を採用する.
その乗車券の入力区間のみを使い,乗車しない区間においては権利放棄をする.
券面表示の発駅以外から乗車することを内方乗車,券面表示の着駅以外で下車することを途中下車と呼ぶ.
ここでは,分割することは考えずに入力された経路を内包し,最も安く乗車することができる1枚の乗車券が出力される.

この運賃計算方法により,1枚の乗車券と分割乗車券のいずれを考えたときでも,経路を外方へ伸ばしたときに運賃が安くなることは無い.
よって,運賃は運賃計算キロに対して単調非減少であることが示せる.

\section{分割乗車券プログラム}

本研究では,運賃が運賃計算キロに対して単調非減少であるという前提に基づき,探索対象を単純パスに限定する.
経路に閉路が含まれる場合,その閉路を取り除いた経路の総距離は必ず元の経路よりも短くなるか等しくなる.
運賃の単調非減少性より,閉路を含む経路が閉路を含まない経路よりも安くなることはない.
同様に,経路の一部を折り返すような経路については,折り返し地点で乗車券を分割する必要があるが,重複区間を取り除いた単純な経路と比較して運賃が安くなることはない.
したがって,最安解となる経路の候補は単純パスの中に必ず存在する.

提案手法の手順は以下の通りである.

\begin{enumerate}
    \item Dijkstra's algorithmを使い,入力された発駅と着駅の間で運賃計算キロが最短となるような経路を1つ選ぶ.
    \item (その経路のキロ単価)÷(最安のキロ単価)倍までの経路をYen's Algorithmを使って候補として分割乗車券の計算を行う.
    \item その中で最安となる分割乗車券を全て出力する.
\end{enumerate}

%%%%%%%%%%%%%%%%%%%%%
% 4章
%%%%%%%%%%%%%%%%%%%%%
\chapter{システムの実装} \label{chapter:implementation}

\section{開発環境と使用技術}
\begin{itemize}
    \item TypeScript(Node.js)
\end{itemize}

\section{運賃・経路データの構築}
以下のデータをそれぞれプログラムへ読み込む.
なお,営業キロと賃率換算キロ及び擬制キロはJR時刻表\cite{jikoku}のデータを利用している.

\begin{itemize}
    \item 入力用データリスト
    \begin{itemize}
        \item 駅名カナデータ,路線駅データ,乗換データ
    \end{itemize}
    \item 特定市内データ
    \begin{itemize}
        \item 名前(例:札幌市内),駅の集合
    \end{itemize}
    \item 経由印字データリスト
    \begin{itemize}
        \item 路線名に紐づく経由印字
    \end{itemize}
    \item セグメントデータリスト
    \begin{itemize}
        \item 営業キロ,換算キロ,旅客会社
    \end{itemize}
    \item 特定運賃リスト
    \item 経路の置き換えリスト
    \item 山手線内データリスト
\end{itemize}

\section{主要モジュールの実装}

\subsection{運賃計算プログラム}
分割乗車券プログラムから呼び出されるプログラムであり,発駅と着駅の他に経路も指定する必要がある.
入力として受け取った発駅から着駅までの経路を含む乗車券のうち最安のものを返す.
ここでは,分割することは考えずに1枚の乗車券として出力する.
ただし,このときに運賃計算経路を外方に伸ばした時に運賃が安くなる場合は乗車券の区間と運賃を伸ばした経路に置き換え,増運賃区間にまたがる場合や鉄道バリアフリー料金も加えた運賃を返す.

\subsection{分割乗車券プログラム}
本システムの中核となる処理であり,以下の手順で実行される.
\begin{enumerate}
    \item \textbf{基準経路の探索:}\\
    Dijkstra's algorithmを用い,発駅から着駅までの運賃計算キロが最短となる経路を探索する.
    これを探索の基準経路とする.
    \item \textbf{候補経路の列挙:}\\
    Yen's Algorithmを用い,第$K$最短経路を順次探索して候補経路リストに追加する.
    なお,探索の打ち切り条件は,第3章で述べた論理に基づき,(最短経路のキロ単価) $\div$ (運賃の最安キロ単価である7.05) を超える距離となった時点とする.
    \item \textbf{各候補経路における最適分割の計算:}\\
    列挙された各候補経路について,個別に最安の分割パターンを算出する.
    具体的には,経路上の各駅をノードとし,任意の2駅間の運賃をエッジの重みとする有向グラフを仮想的に構築し,その上での最短経路問題を解くことで,その経路における最安運賃を確定させる.
    \item \textbf{解の出力:}\\
    全ての候補経路の中で最も運賃が安い分割パターンと経路を出力する.
\end{enumerate}

%%%%%%%%%%%%%%%%%%%%%
% 5章
%%%%%%%%%%%%%%%%%%%%%
\chapter{評価実験と考察} \label{chapter:evaluation}

\section{実験概要}
評価に用いた計算機のスペック:MacBook Pro, Apple M4, 16GB

\section{計算結果の正当性評価}

\section{性能評価}
計算時間を計測する.

\section{考察}
計算時間がかかったケースがあれば,その原因を考える.

%%%%%%%%%%%%%%%%%%%%%
% 6章
%%%%%%%%%%%%%%%%%%%%%
\chapter{結論} \label{chapter:conclusion}
\section{本研究の総括}
本研究では,JR分割乗車券の最適解を探索するプログラムをDijkstra's algorithmを用いて開発した.

\section{将来の展望と今後の課題}
\begin{itemize}
    \item 展望:新幹線や特急料金を含めた最適化,Webサービスとしての一般公開,モバイルアプリ化など.
    \item 課題:運賃改定への自動追従システムの構築,計算速度の更なる向上,各種割引への対応.
\end{itemize}

%%%%%%%%%%%%%%%%%%%%%
% 謝辞
%%%%%%%%%%%%%%%%%%%%%
\syaji
\par
本論文について貴重な御助言を頂いた指導教員や研究室のメンバー方には心より感謝致します.

%%%%%%%%%%%%%%%%%%%%%
% 参考文献
%%%%%%%%%%%%%%%%%%%%%
\begin{thebibliography}{99}
\addcontentsline{toc}{chapter}{参考文献}

\bibitem{dijkstra}
E.W. Dijkstra, A note on two problems in connexion with graphs, Numerische Mathematik, vol.~1, pp.~269--271, 1959.

\bibitem{yen}
Jin Y. Yen, Finding the K Shortest Loopless Paths in a Network, Management Science, vol.~17, no.~11, pp.~712--716, 1971.

\bibitem{oba}
oba,乗車券分割プログラム,入手先〈\url{https://bunkatsu.jp/}〉(参照2025-12-25).

\bibitem{ryokaku}
JR東日本,旅客営業規則,入手先〈\url{https://www.jreast.co.jp/ryokaku/}〉(参照2025-12-25).

\bibitem{morita}
森田 隼史, 池上 敦子, 菊地 丞, 山口 拓真, 中山 利宏, 大倉 元宏, 鉄道運賃計算アルゴリズム : Suica/PASMO利用可能範囲のJR東日本510駅の運賃を対象とした場合, 日本オペレーションズ・リサーチ学会和文論文誌, vol.~54, pp.~1--22, 2011.

\bibitem{hirata}
平田 直也, 中桐 斉之, スマートフォンに特化した乗車券分割アプリの開発, 第78回全国大会講演論文集, vol.~2016, no.~1, pp.~387--388, 2016.

\bibitem{jikoku}
時刻表編集部,JR時刻表 2025年10月号,交通新聞社,2025.

\end{thebibliography}

%%%%%%%%%%%%%%%%%%%%%
% 付録
%%%%%%%%%%%%%%%%%%%%%
\appendix
\def\labelenumi{(\theenumi)}
\chapter{運賃計算方法と特例}

片道乗車券の運賃計算する際の経路補正に関する規則のみを抜粋して掲載する.規則の全文については、参考文献\cite{ryokaku}を参照されたい.
\\\\
\noindent\textbf{(営業キロ)}\\
\noindent\textbf{第14条}\\
旅客運賃・料金の計算その他の旅客運送の条件をキロメートルをもって定める場合は、別に定める場合を除き、営業キロによる。\\

\noindent\textbf{2}\\
前条の営業キロは、旅客の乗車する発着区間に対する駅間のキロ数による。\\

\noindent\textbf{(運賃計算キロ)}\\
\noindent\textbf{第14条の2}\\
前条の規定によるほか、幹線と地方交通線を連続して乗車する場合(幹線と地方交通線の中間に当社と通過連絡運輸を行う鉄道・軌道・航路又は自動車線が介在する場合で、これらを通じて連続乗車するときを含む。以下同じ。)の旅客運賃を計算するときは、旅客の乗車する発着区間のうち、地方交通線の乗車区間に対する営業キロを賃率比に応じて換算したもの(以下、北海道旅客鉄道株式会社、東日本旅客鉄道株式会社、東海旅客鉄道株式会社及び西日本旅客鉄道株式会社にあっては「賃率換算キロ」、四国旅客鉄道株式会社及び九州旅客鉄道株式会社にあっては「擬制キロ」という。)と幹線の乗車区間に対する営業キロを合算したもの(以下「運賃計算キロ」という。)による。\\

\noindent\textbf{2}\\
前項の賃率換算キロ及び擬制キロは、別に定めるものとし、地方交通線の乗車区間に対する営業キロに、第77条の5に規定する地方交通線の第1地帯賃率を第77条に規定する幹線の第1地帯賃率で除した値を乗じて得たもの(小数点以下1位未満の端数があるときはこれを四捨五入する。)とする。\\
ただし、北海道旅客鉄道会社線内にあっては、地方交通線の乗車区間に対する営業キロに、第77条の6に規定する地方交通線の第1地帯賃率を第77条の2に規定する幹線の第1地帯賃率で除した値を乗じて得たものとする。\\

\noindent\textbf{(擬制キロ)}\\
\noindent\textbf{第14条の3}\\
第14条の規定にかかわらず、四国旅客鉄道会社線又は九州旅客鉄道会社線の地方交通線内各駅相互間を乗車する場合の旅客運賃を計算するときは、前条第1項に定める擬制キロによる。\\

\noindent\textbf{(他の旅客鉄道会社線を通じて連続乗車する場合の営業キロ、賃率換算キロ、擬制キロ又は運賃計算キロの通算)}\\
\noindent\textbf{第14条の4}\\
当社線と他の旅客鉄道会社線を通じて連続乗車する場合の営業キロ、賃率換算キロ、擬制キロ又は運賃計算キロは、旅客の乗車区間に対し、第14条又は第14条の2の規定を適用して計算したものによる。\\

\noindent\textbf{2}\\
前項の規定による営業キロ、賃率換算キロ、擬制キロ又は運賃計算キロは、旅客運賃・料金の計算その他この規則に定める取扱いをする場合に適用する。\\

\noindent\textbf{(常磐線北千住・綾瀬間相互発着となる旅客の取扱い)}\\
\noindent\textbf{第16条の5}\\
常磐線北千住・綾瀬間相互発着となる旅客に対しては、乗車券類の発売を行わないものとする。\\

\noindent\textbf{(普通乗車券の発売)}\\
\noindent\textbf{第26条}\\
旅客が、列車に乗車する場合は、次の各号に定めるところにより、片道乗車券、往復乗車券又は連続乗車券を発売する。
\begin{enumerate}
    \item 片道乗車券\\
    普通旅客運賃計算経路の連続した区間を片道1回乗車(以下「片道乗車」という。)する場合に発売する。ただし、第68条第4項の規定により営業キロ、擬制キロ又は運賃計算キロを打ち切って計算する場合は、当該打切りとなる駅までの区間のものに限り発売する。
\end{enumerate}

\noindent\textbf{(普通乗車券の発売方)}\\
\noindent\textbf{第26条の2}\\
次の各号に掲げる場合は、前条及び第68条第4項の規定により、それぞれ片道乗車券又は連続乗車券を発売する。
\begin{enumerate}
    \item 環状線一周となる経路の場合は、片道乗車券を発売する。
\end{enumerate}

\noindent\textbf{(旅客運賃・料金計算上の経路等)}\\
\noindent\textbf{第67条}\\
旅客運賃・料金は、旅客の実際乗車する経路及び発着の順序によって計算する。

\noindent\textbf{(旅客運賃・料金計算上の営業キロ等の計算方)}\\
\noindent\textbf{第68条}\\
営業キロ又は擬制キロを使用して旅客運賃を計算する場合は、別に定める場合を除いて、次の各号により営業キロ又は擬制キロを通算して計算する。
\begin{enumerate}
    \item 営業キロ又は擬制キロは、同ー方向に連続する場合に限り、これを通算する。
\end{enumerate}

\noindent\textbf{2}\\
前項の規定は、運賃計算キロを使用して幹線と地方交通線を連続して乗車するときの旅客運賃を計算する場合に準用する。\\

\noindent\textbf{4}\\
前各項の規定により、旅客運賃・料金を計算する場合で次の各号の1に該当するときは、当該各号に定めるところによって計算する。
\begin{enumerate}
    \item 計算経路が環状線1周となる場合は、環状線1周となる駅の前後の区間の営業キロ、擬制キロ又は運賃計算キロを打ち切って計算する。
    \item 計算経路の一部若しくは全部が復乗となる場合は、折返しとなる駅の前後の区間の営業キロ、擬制キロ又は運賃計算キロを打ち切って計算する。
\end{enumerate}

\noindent\textbf{(特定区間における旅客運賃・料金計算の営業キロ又は運賃計算キロ)}\\
\noindent\textbf{第69条}\\
第67条の規定にかかわらず、次の各号に掲げる区間の普通旅客運賃・料金は、その旅客運賃・料金計算経路が当該各号末尾のかっこ内の両線路にまたがる場合を除いて、○印の経路の営業キロ(第9号については運賃計算キロ。ただし、岩国・櫛ヶ浜間相互発着の場合にあっては営業キロ)によって計算する。この場合、各号の区間内については、経路の指定を行わない。

\begin{enumerate}
    \item 大沼以遠(仁山方面)の各駅と、森以遠(石倉方面)の各駅との相互間\\
        \qquad    
        $
            \Bigg(
                \begin{tabular}{l}
                     東森駅経由函館本線\\
                    ○大沼公園駅経由函館本線
                \end{tabular}
            \Bigg)
        $
    
    \item 日暮里以遠(鶯谷又は三河島方面)の各駅と、赤羽以遠(川口、北赤羽又は十条方面)の各駅との相互間\\
        \qquad    
        $
            \Bigg(
                \begin{tabular}{l}
                     尾久経由東北本線\\
                    ○王子経由東北本線
                \end{tabular}
            \Bigg)
        $

    \item 赤羽以遠(尾久、東十条又は十条方面)の各駅と、大宮以遠(土呂、宮原又は日進方面)の各駅との相互間\\
        \qquad    
        $
            \Bigg(
                \begin{tabular}{l}
                     戸田公園・与野本町経由東北本線\\
                    ○川口・浦和経由東北本線
                \end{tabular}
            \Bigg)
        $

    \item 品川以遠(高輪ゲートウェイ又は大崎方面)の各駅と、鶴見以遠(新子安、国道又は羽沢横浜国大方面)の各駅との相互間\\
        \qquad    
        $
            \Bigg(
                \begin{tabular}{l}
                     西大井経由東海道本線\\
                    ○大井町経由東海道本線
                \end{tabular}
            \Bigg)
        $

    \item 東京以遠(有楽町又は神田方面)の各駅と、蘇我以遠(鎌取又は浜野方面)の各駅との相互間\\
        \qquad    
        $
            \Bigg(
                \begin{tabular}{l}
                     京葉線\\
                    ○総武本線・外房線
                \end{tabular}
            \Bigg)
        $

    \item 山科以遠(京都方面)の各駅と、近江塩津以遠(新疋田方面)の各駅との相互間\\
        \qquad    
        $
            \Bigg(
                \begin{tabular}{l}
                     東海道本線・北陸本線\\
                    ○湖西線
                \end{tabular}
            \Bigg)
        $

    \item 大阪以遠(塚本又は新大阪方面)の各駅と、天王寺以遠(東部市場前又は美章園方面)の各駅との相互間\\
        \qquad    
        $
            \Bigg(
                \begin{tabular}{l}
                     福島経由大阪環状線\\
                    ○天満経由大阪環状線
                \end{tabular}
            \Bigg)
        $

    \item 三原以遠(糸崎方面)の各駅と、海田市以遠(向洋方面)の各駅との相互間\\
        \qquad    
        $
            \Bigg(
                \begin{tabular}{l}
                     呉線\\
                    ○山陽本線
                \end{tabular}
            \Bigg)
        $

    \item 岩国以遠(和木方面)の各駅と、櫛ヶ浜以遠(徳山方面)の各駅との相互間\\
        \qquad    
        $
            \Bigg(
                \begin{tabular}{l}
                     山陽本線\\
                    ○岩徳線
                \end{tabular}
            \Bigg)
        $

\end{enumerate}

\noindent\textbf{第70条}\\
第67条の規定にかかわらず、旅客が次に掲げる図の太線区間を通過する場合の普通旅客運賃・料金は太線区間内の最も短い営業キロによって計算する。この場合、太線内は、経路の指定を行わない。\\
\includegraphics[scale=0.9]{70.png}

\noindent\textbf{2}\\
蘇我以遠(鎌取又は浜野方面)の各駅と前条第1項第5号に掲げるいずれかの経路を経由して前項に掲げる図の太線区間を大久保以遠(東中野方面)、三河島以遠(南千住方面)、川口以遠(西川口方面)又は北赤羽以遠(浮間舟渡方面)へ通過する場合の普通旅客運賃・料金は、第67条及び前条第1項第5号の規定にかかわらず、外房線蘇我・千葉間、総武本線千葉・錦糸町間及び前項に掲げる図の太線区間内の最も短い経路の営業キロによって計算する。

\noindent\textbf{(幹線内相互発着の大人片道普通旅客運賃)}\\
\noindent\textbf{第77条}\\
幹線内相互発着となる場合の大人片道普通旅客運賃は、次の各号により計算した額を合計した額とする。\\
ただし、北海道旅客鉄道会社線、四国旅客鉄道会社線又は九州旅客鉄道会社線内発又は着若しくは通過となる場合を除く。
\begin{enumerate}
    \item 発着区間の営業キロを次の営業キロに従って区分し、これに各その営業キロに対する賃率を乗じた額を合計した額。この場合、発着区間の営業キロが100キロメートル以下のときは、10円未満のは数を10円単位に切り上げた額とし、100キロメートルを超えるときは、50円未満のは数を切り捨てて、又は50円以上のは数を切り上げてそれぞれ100円単位とした額とする。\\
        \begin{tabular}{|l|l|l|} \hline
            \makecell[tl]{300キロメートル以下の営業キロ \\ (第1地帯)} & 
                1キロメートルにつき & 
                16円20銭 \\ \hline
            
            \makecell[tl]{300キロメートルを超え、\\600キロメートル以下の営業キロ \\ (第2地帯)} & 
                1キロメートルにつき & 
                12円85銭 \\ \hline
            
            \makecell[tl]{600キロメートルを超える営業キロ \\ (第3地帯)} & 
                1キロメートルにつき & 
                7円05銭 \\ \hline
        \end{tabular}
    \item 前号の規定により計算した額に100分の10を乗じ10円未満のは数を円位において四捨五入して10円単位とした額(以下この方法を「四捨五入」という。)
\end{enumerate}

\noindent\textbf{2}\\
前項の規定によるほか、幹線内相互発着の大人片道普通旅客運賃は、次の各号に定める営業キロのものを適用する。
\begin{enumerate}
    \item 11キロメートルから50キロメートルまで\\
        11キロメートルから5キロメートルごとに区分し、11キロメートルから15キロメートルまでは13キロメートルとし、16キロメートル以上は、これに1区分を増すごとに5キロメートルを加えた営業キロとする。
    \item 51キロメートルから100キロメートルまで\\
        51キロメートルから10キロメートルごとに区分し、51キロメートルから60キロメートルまでは55キロメートルとし、61キロメートル以上は、これに1区分を増すごとに10キロメートルを加えた営業キロとする。
    \item 101キロメートルから600キロメートルまで\\
        101キロメートルから20キロメートルごとに区分し、101キロメートルから120キロメートルまでは110キロメートルとし、121キロメートル以上は、これに1区分を増すごとに20キロメートルを加えた営業キロとする。
    \item 601キロメートル以上\\
        601キロメートルから40キロメートルごとに区分し、601キロメートルから640キロメートルまでは620キロメートルとし、641キロメートル以上は、これに1区分を増すごとに40キロメートルを加えた営業キロとする。
\end{enumerate}

\noindent\textbf{(北海道旅客鉄道会社内の幹線内相互発着の大人片道普通旅客運賃)}\\
\noindent\textbf{第77条の2}\\
北海道旅客鉄道会社内の幹線内相互発着となる場合の大人片道普通旅客運賃は、次の各号に定めるとおりとする。
\begin{enumerate}
    \item 営業キロが11キロメートルから100キロメートルまでの場合\\
        \noindent
        \begin{tabularx}{\linewidth}{|X|X|} \hline
            \multicolumn{1}{|c|}{営業キロの区間} &
            \multicolumn{1}{c|}{大人片道普通旅客運賃} \\ \hline
            15キロメートルまで & 360円 \\ \hline
            15キロメートルを超え、25キロメートルまで &
            5キロメートルまでを増すごとに110円加算 \\ \hline
            25キロメートルを超え、30キロメートルまで & 680円 \\ \hline
            30キロメートルを超え、35キロメートルまで & 800円 \\ \hline
            35キロメートルを超え、45キロメートルまで &
            5キロメートルまでを増すごとに120円加算 \\ \hline
            45キロメートルを超え、50キロメートルまで & 1,210円 \\ \hline
            50キロメートルを超え、60キロメートルまで & 1,380円 \\ \hline
            60キロメートルを超え、70キロメートルまで & 1,590円 \\ \hline
            70キロメートルを超え、80キロメートルまで & 1,800円 \\ \hline
            80キロメートルを超え、90キロメートルまで & 2,020円 \\ \hline
            90キロメートルを超え、100キロメートルまで & 2,240円 \\ \hline
        \end{tabularx}

    \item 営業キロが100キロメートルを超える場合
        発着区間の営業キロを次の営業キロに従って区分し,
        各その営業キロに対する賃率により,
        第77条第1項並びに同条第2項第3号及び第4号の規定を
        適用して計算した額とする。
        \newline
        \noindent
        \begin{tabularx}{\linewidth}{|X|l|l|} \hline
            200キロメートル以下の営業キロ\newline(第1地帯) &
            1キロメートルにつき & 21円16銭 \\ \hline
            200キロメートルを超え、300キロメートル以下の営業キロ\newline(第2地帯) &
            1キロメートルにつき & 16円36銭 \\ \hline
            300キロメートルを超え、600キロメートル以下の営業キロ\newline(第3地帯) &
            1キロメートルにつき & 12円83銭 \\ \hline
            600キロメートルを超える営業キロ\newline(第4地帯) &
            1キロメートルにつき & 7円05銭 \\ \hline
        \end{tabularx}
\end{enumerate}

\noindent\textbf{2}\\
前項の規定にかかわらず、別表第2号イに定める営業キロの区間の大人片道普通旅客運賃については、同表に定めるところにより特定の額とする。

\noindent\textbf{(四国旅客鉄道会社内の幹線内相互発着の大人片道普通旅客運賃)}\\
\noindent\textbf{第77条の3}\\
四国旅客鉄道会社内の幹線内相互発着となる場合の大人片道普通旅客運賃は、次の各号に定めるとおりとする。
    \begin{enumerate}
    \item 営業キロが11キロメートルから100キロメートルまでの場合
        \newline
        \noindent
        \begin{tabularx}{\linewidth}{|X|X|} \hline
            \multicolumn{1}{|c|}{営業キロの区間} &
            \multicolumn{1}{c|}{大人片道普通旅客運賃} \\ \hline
            15キロメートルまで & 330円 \\ \hline
            15キロメートルを超え、30キロメートルまで &
            5キロメートルまでを増すごとに100円加算 \\ \hline
            30キロメートルを超え、35キロメートルまで & 740円 \\ \hline
            35キロメートルを超え、40キロメートルまで & 850円 \\ \hline
            40キロメートルを超え、45キロメートルまで & 980円 \\ \hline
            45キロメートルを超え、50キロメートルまで & 1,080円 \\ \hline
            50キロメートルを超え、60キロメートルまで & 1,240円 \\ \hline
            60キロメートルを超え、70キロメートルまで & 1,430円 \\ \hline
            70キロメートルを超え、80キロメートルまで & 1,640円 \\ \hline
            80キロメートルを超え、90キロメートルまで & 1,830円 \\ \hline
            90キロメートルを超え、100キロメートルまで & 2,010円 \\ \hline
        \end{tabularx}

    \item 営業キロが100キロメートルを超える場合
        発着区間の営業キロを次の営業キロに従って区分し,
        各その営業キロに対する賃率により,
        第77条第1項並びに同条第2項第3号及び第4号の規定を
        適用して計算した額とする。
        \newline
        \noindent
        \begin{tabularx}{\linewidth}{|X|l|l|} \hline
            200キロメートル以下の営業キロ\newline(第1地帯) &
            1キロメートルにつき & 19円20銭 \\ \hline
            200キロメートルを超え、300キロメートル以下の営業キロ\newline(第2地帯) &
            1キロメートルにつき & 16円20銭 \\ \hline
            300キロメートルを超え、600キロメートル以下の営業キロ\newline(第3地帯) &
            1キロメートルにつき & 12円85銭 \\ \hline
            600キロメートルを超える営業キロ\newline(第4地帯) &
            1キロメートルにつき & 7円05銭 \\ \hline
        \end{tabularx}
\end{enumerate}


\noindent\textbf{(鉄道駅バリアフリー料金)}\\
\noindent\textbf{第140条}\\
次の各号に掲げる区間内相互発着となる区間に乗車する場合は、鉄道駅バリアフリー料金を収受する。
\begin{enumerate}
    \item 第78条第2項第1号に定める東京附近における電車特定区間及び第80条の規定を適用する区間(同条第1項第1号から第4号の区間にかかるものに限る。)
    \item 第78条第2項第2号に定める大阪附近における電車特定区間及び第80条の規定を適用する区間(同条第1項第5号から第14号及び同条第2項の区間にかかるものに限る。)
    \item 東海道本線(新幹線)中豊橋・岐阜羽島間、東海道本線中豊橋・大垣間、武豊線、中央本線中多治見・名古屋間、関西本線中名古屋・四日市間(ただし、対象区間のみを経由して乗車する場合に限る。)
\end{enumerate}

\noindent\textbf{2}\\
前項の規定により収受する鉄道駅バリアフリー料金は、次の各号に定めるとおりとする。
\begin{enumerate}
    \item 前項第1号に掲げる区間内相互発着となる区間に乗車する場合\\
    イ 大人片道普通旅客運賃とあわせ収受する額\\
      片道乗車あたり10円
    \item 前項第2号及び第3号に掲げる区間内相互発着となる区間に乗車する場合\\
    イ 大人片道普通旅客運賃とあわせ収受する額\\
      片道乗車あたり10円
\end{enumerate}



\chapter{実装したプログラムのコード}

\end{document}