\documentclass[a4j,12pt]{jreport}
%\documentclass{jreport}
\usepackage[dvipdfmx]{graphicx}
\usepackage{booktabs}
\usepackage{amsmath,amssymb}
\usepackage{here}
\usepackage{algorithm}
\usepackage{algpseudocode}
\usepackage{hhline}
\usepackage{makecell}
\usepackage{tabularx}
\usepackage[hang,small,bf]{caption}
\usepackage[subrefformat=parens]{subcaption}
\usepackage{url}
\captionsetup{compatibility=false}

\def\syaji{ \chapter*{謝辞} \addcontentsline{toc}{chapter}{謝辞}}
\renewcommand{\bibname}{参考文献}
\setlength{\textheight}{\paperheight}
\setlength{\topmargin}{4.6mm}
\addtolength{\topmargin}{-\headheight}
\addtolength{\topmargin}{-\headsep}
\addtolength{\topmargin}{-\headheight}
\addtolength{\textheight}{-60mm}

\setlength{\textwidth}{\paperwidth}
\setlength{\oddsidemargin}{-0.4mm}
\setlength{\evensidemargin}{-0.4mm}
\addtolength{\textwidth}{-50mm}

\begin{document}

%%%%%%%%%%%%%%%%%%%%%
% 表紙
%%%%%%%%%%%%%%%%%%%%%
\thispagestyle{empty}
\begin{center}
  \begin{Large}
    \vspace*{0.7cm}
    {\large 卒業研究論文}\\
    \vspace*{2.5cm}
    {\LARGE\bf JR最安分割乗車券探索システムの開発}\\
    \vspace*{7.5cm}
    東條 涼平\\
    学籍番号\hspace*{1zw}22D8102015D\\
    \vspace*{2.5cm}
    中央大学理工学部情報工学科\hspace*{1zw} アルゴリズム理論基礎研究室\\
    \vspace*{3.0cm}
    2026年3月\\
  \end{Large}
\end{center}


%%%%%%%%%%%%%%%%%%%%%
% 概要
%%%%%%%%%%%%%%%%%%%%%
\newpage
\renewcommand{\baselinestretch}{1.25} \selectfont
\pagenumbering{roman}


\begin{center} {\large \bf{概 要}} \end{center}

JR線を利用する際,安く移動をしようとするとき,通しで購入する乗車券の運賃よりも乗車区間を途中で区切って複数枚の乗車券を購入するほうが安い場合がある.
この区間を分けて購入した乗車券を分割乗車券という.
従来,任意の駅間において分割乗車券を考慮したときに最安解を出すためには,最適解のルートとして考えられる経路の全てで分割乗車券の運賃を比較する必要があったが,このとき分割乗車券の経路が最短経路でない場合もあるため,複数の経路を試す必要があった.
そこで,Dijkstra's algorithm\cite{dijkstra}とYen's Algorithm\cite{yen}を使って最安となりうる経路の候補を出し,それぞれに対して分割経路を計算をして,それらの結果をもとに最安解を出力することにより,経路を指定せずに発駅と着駅の入力のみで最適解を出すプログラムを作成した.

\vspace{1zw} \noindent
{\bf キーワード: }分割乗車券,経路探索,JR運賃計算,グラフ理論

%%%%%%%%%%%%%%%%%%%%%
% 目次
%%%%%%%%%%%%%%%%%%%%%
\tableofcontents


\newpage
\pagenumbering{arabic}

%%%%%%%%%%%%%%%%%%%%%
% 1章
%%%%%%%%%%%%%%%%%%%%%
\chapter{序論} \label{chapter:intro}

\section{研究の背景}
JRでは営業キロを一定の幅に区切って運賃を設定しているため,乗車券の区間を分割して購入したほうが安くなる場合がある.
区間を分けて乗車券を購入することを分割乗車券といい,2000年以前からインターネット上で知られていた.
発駅と着駅のほかに経路を指定した場合であれば,最安解を出力する分割乗車券プログラム\cite{oba}が既に存在する.
多くの場合では,発駅から着駅までを結ぶ最短経路上に分割駅が存在するが,都心のような路線が狭い範囲で路線が密集している場合など,最短経路上に分割駅が無い場合もある.
そのため,旅客が最安解である経路を特定することは困難である.

\section{研究の対象}
今回の研究では,大人の片道乗車券を対象とする.
また,新幹線を使った経路では乗車券の他に特急券が必要となるため今回の研究では在来線のみを探索の対象とする.
BRT線においても,発売に制限があるため探索の対象外とする.
片道乗車券の運賃計算には多くの例外があり複雑である.
例えば,同一の区間であっても複数の正規運賃が存在することもある.
この研究で扱う乗車券の運賃と経路は,運賃計算の手順が記載されている旅客営業規則に則って計算をして,駅の窓口や自動券売機で実際に購入できるものを扱い,ICカード運賃は考慮しない.

\section{関連研究・従来の手法}
分割乗車券プログラムをWebアプリとして公開しているサイトが存在する.このサイトでは,発駅と着駅の他に経路を指定する必要があり,その経路上で最安分割乗車券を探索する.
また,分割乗車券に特化した先行研究として,平田ら\cite{hirata}によるスマートフォンに特化した乗車券分割アプリの事例がある.
しかし,平田らの手法では最短経路上で分割回数が1回に限定されているという制約があり,探索エリアも関西の一部エリアのみである.

\section{研究の目的と意義}
旅客らに実用的な節約手段を提供でき,複雑な制約を持つ大規模グラフ問題に対する効率的な解法を高速な計算時間で実装する.

%%%%%%%%%%%%%%%%%%%%%
% 2章
%%%%%%%%%%%%%%%%%%%%%
\chapter{JR運賃制度と問題の定義} \label{chapter:fare_system}

\section{JR運賃計算の基礎}
JRは6つの旅客鉄道会社などから構成される.
この研究では旅客鉄道会社のみを扱い,これ以降は通称を使う.
表\ref{tab:companies}に各社の通称を示す.

\begin{table}[H]
\centering
\caption{旅客鉄道会社一覧}
\label{tab:companies}
\begin{tabular}{|c|c|} \hline
会社名 & 通称 \\ \hline
北海道旅客鉄道 & JR北海道 \\ \hline
東日本旅客鉄道 & JR東日本 \\ \hline
東海旅客鉄道 & JR東海 \\ \hline
西日本旅客鉄道 & JR西日本 \\ \hline
四国旅客鉄道 & JR四国 \\ \hline
九州旅客鉄道 & JR九州 \\ \hline
\end{tabular}
\end{table}

また,JR東日本・JR東海・JR西日本をまとめて本州3社と呼ぶ.各旅客鉄道会社に旅客営業規則が定められているが,それぞれに内容の差異は無い.
この研究では,JR東日本の旅客営業規則\cite{ryokaku}を参照する.

まずは,片道乗車券の定義について解説をする.片道乗車券における経路は重複してはならず,環状線を超える経路であってはならない.
ここで,環状線を超えるとは,発駅から一駅ずつ経路を進めて行った場合に閉路ができたらその駅までの乗車券は片道乗車券として発売可能であり,それより先に経路を伸ばした場合は発売ができないことを示す.
つまり,発着駅が同じとなる円状の経路は発売可能である.

次に,運賃計算をする手順を以下に示す.
まず,JRの路線は「幹線」と「地方交通線」のいずれかに分類されており,各駅間には0.1km単位で営業キロが設定されている.
運賃は,乗車区間の営業キロの合計に基づき,対応する距離区分(キロ地帯)や賃率を適用して算出される.
本州3社の幹線内相互発着の場合における大人片道普通旅客運賃の計算手順は旅客営業規則 第77条に定められている.
同様にして,本州3社の地方交通線内相互発着の場合における大人片道普通旅客運賃は旅客営業規則 第77条の5に定められている.
よって,営業キロが200kmまでの運賃表は表\ref{tab:fare_comparison}の通りとなる.

また,幹線と地方交通線を連続して乗車する場合は,地方交通線の乗車区間に対する営業キロにおおよそ1.1倍した値が用いられる.
これを,JR北海道,JR東日本,JR東海,JR西日本では「賃率換算キロ」といい,JR四国,JR九州では「擬制キロ」という.
これらは,地方交通線の各駅間に設定されており,幹線の乗車区間に対する営業キロと地方交通線の乗車区間に対する「賃率換算キロ」及び「擬制キロ」を合算したものを「運賃計算キロ」という.
この運賃計算キロを幹線と営業キロとして,対応する距離区分(キロ地帯)や賃率を適用して算出される.

\begin{table}[htbp]
  \caption{本州3社の幹線と地方交通線における運賃の比較}
  \label{tab:fare_comparison}
  \centering
  \begin{minipage}[t]{0.48\textwidth}
    \centering
    \caption*{本州3社の幹線の運賃表}
    \begin{tabular}{cr} \toprule
      営業キロ (km) & 運賃 (円) \\ \midrule
      $\ \ \ \ \ \ \sim \ \ 3.0$   & 150 \\
      $\ \ 3.1 \sim \ \ 6.0$   & 190 \\
      $\ \ 6.1 \sim \ 10.0$  & 200 \\
      $\ 10.1 \sim \ 15.0$ & 240 \\
      $\ 15.1 \sim \ 20.0$ & 330 \\
      $\ 20.1 \sim \ 25.0$ & 420 \\
      $\ 25.1 \sim \ 30.0$ & 510 \\
      $\ 30.1 \sim \ 35.0$ & 590 \\
      $\ 35.1 \sim \ 40.0$ & 680 \\
      $\ 40.1 \sim \ 45.0$ & 770 \\
      $\ 45.1 \sim \ 50.0$ & 860 \\
      $\ 50.1 \sim \ 60.0$ & 990 \\
      $\ 60.1 \sim \ 70.0$ & 1,170 \\
      $\ 70.1 \sim \ 80.0$ & 1,340 \\
      $\ 80.1 \sim \ 90.0$ & 1,520 \\
      $\ 90.1 \sim 100.0$ & 1,690 \\
      $100.1 \sim 120.0$ & 1,980 \\
      $120.1 \sim 140.0$ & 2,310 \\
      $140.1 \sim 160.0$ & 2,640 \\
      $160.1 \sim 180.0$ & 3,080 \\
      $180.1 \sim 200.0$ & 3,410 \\ \bottomrule
    \end{tabular}
  \end{minipage}
  \hfill
  \begin{minipage}[t]{0.48\textwidth}
    \centering
    \caption*{本州3社の地方交通線の運賃表}
    \begin{tabular}{cr} \toprule
      営業キロ (km) & 運賃 (円) \\ \midrule
      $\ \ \ \ \ \ \sim \ \ 3.0$   & 150 \\
      $\ \ 3.1 \sim \ \ 6.0$   & 190 \\
      $\ \ 6.1 \sim \ 10.0$  & 210 \\
      $\ 10.1 \sim \ 15.0$ & 240 \\
      $\ 15.1 \sim \ 20.0$ & 330 \\
      $\ 20.1 \sim \ 23.0$ & 420 \\
      $\ 23.1 \sim \ 28.0$ & 510 \\
      $\ 28.1 \sim \ 32.0$ & 590 \\
      $\ 32.1 \sim \ 37.0$ & 680 \\
      $\ 37.1 \sim \ 41.0$ & 770 \\
      $\ 41.1 \sim \ 46.0$ & 860 \\
      $\ 46.1 \sim \ 55.0$ & 990 \\
      $\ 55.1 \sim \ 64.0$ & 1,170 \\
      $\ 64.1 \sim \ 73.0$ & 1,340 \\
      $\ 73.1 \sim \ 82.0$ & 1,520 \\
      $\ 82.1 \sim \ 91.0$ & 1,690 \\
      $\ 91.1 \sim 100.0$ & 1,880 \\
      $100.1 \sim 110.0$ & 1,980 \\
      $110.1 \sim 128.0$ & 2,310 \\
      $128.1 \sim 146.0$ & 2,640 \\
      $146.1 \sim 164.0$ & 3,080 \\
      $164.1 \sim 182.0$ & 3,410 \\
      $182.1 \sim 200.0$ & 3,740 \\ \bottomrule
    \end{tabular}
  \end{minipage}
\end{table}

\section{鉄道駅バリアフリー料金について}
旅客営業規則 第140条より,一部区間内を相互発着となるような経路に乗車する場合は,運賃の他に鉄道駅バリアフリー料金が収受される.
今回の研究では,該当の区間相互発着となった場合は鉄道駅バリアフリー料金である10円をあわせた金額を運賃として扱い,実際の切符に印字される金額と出力を一致させる.

\section{経路を外方へ伸ばしたほうが運賃が安い場合}
運賃は,発駅から着駅までの経路区間に対する運賃計算キロで計算される.
しかし,本研究の主目的である「移動コストの最小化」を達成するためには,単に発着駅間の経路に基づいて運賃を計算するだけでは不十分である.
運賃は,乗車経路の営業キロあるいは運賃計算キロに対して単調非減少であるが,川浦\cite{trrc}が指摘するように,特定の条件下では経路を外方へ延伸することで運賃経路の営業キロあるいは運賃計算キロが短くなり運賃が安くなる「逆転現象」が発生するためである.
この逆転現象が起こった場合,その乗車券は入力された乗車区間のみを使い,乗車しない区間においては権利放棄をして,運賃は乗車経路に対して単調非減少となるようにした.
このとき,券面表示の発駅以外から乗車することを内方乗車,券面表示の着駅以外で下車することを前途放棄と呼ぶ.
本システムでは,以下の事例に代表される逆転現象の特例処理について考慮した実装をし,真の最安運賃を導出できるようにした.

\subsection{経路特定区間に関する運賃の逆転現象}
旅客営業規則 第69条で定められている経路特定区間を乗り通す場合は,いずれの経路を利用した場合でも○印の経路で運賃計算が行われる.
このとき,○印のない経路の途中駅からあるいは途中駅までの運賃よりも,経路を外方の分岐駅まで伸ばすことにより,旅客営業規則 第69条を適用させて,運賃経路を○印経由とした方が安い場合がある.
このプログラムでは,経路特定区間○印でない経路上に発着駅があった場合に,その駅を外方の分岐駅まで伸ばした運賃と比較することにより逆転現象が起きないようにした.

\subsection{特定都区市内制度の制度設計に起因する運賃の逆転現象}
旅客営業規則 第86条及び第87条が適用されないような特定都区市内あるいは東京山手線内を発着駅とする経路のうち,旅客営業規則 第86条 及び 第87条 を適用するために,もう一方の駅を外方に伸ばすことで,中心駅からの営業キロが閾値を超えるようにして,運賃を安くすることができる場合がある.
このとき,運賃が安くなった場合は入力された駅ではなく外方の駅までの区間と運賃を出力する.

\section{問題の定式化}
発駅と着駅のみを入力として,その間を最も安く移動するような分割乗車券を出力する.このとき,乗車券の分割数には制限を設けず,最安分割乗車券の候補となる経路は全て探索する.

%%%%%%%%%%%%%%%%%%%%%
% 3章
%%%%%%%%%%%%%%%%%%%%%
\chapter{探索手法} \label{chapter:algorithm}

\section{運賃計算プログラム}
分割乗車券プログラムから呼び出されるプログラムであり,入力では発駅と着駅の他に経路も指定する必要がある.

運賃計算アルゴリズムは以下の通りである.
\begin{itemize}
    \item 並行する他社路線に対抗するため,通常の距離制運賃よりも安く設定された区間である特定運賃区間と経路が一致する場合はその特定運賃を返す.
    \item 入力された経路が,山手線内の駅相互発着の場合,東京附近における電車特定区間内相互発着の場合,大阪附近における電車特定区間内相互発着の場合はそれぞれに対応した賃率により運賃を計算する.
    \item 全ての駅間を旅客会社ごとに分けて,まずはすべての区間を本州3社の運賃体系で計算をする.
    JR北海道,JR四国,JR九州のそれぞれの区間で,会社ごとの運賃から本州3社の運賃の差を加算額として加える.
    また,加算運賃区間にまたがる場合は加算運賃を加えた額を返す.
\end{itemize}

運賃計算をした上で,鉄道バリアフリー料金該当区間相互発着の場合は鉄道バリアフリー料金を加えた額を運賃とする.\\

この運賃計算プログラムでは,旅客営業規則の通りに,入力された経路が計算される.
しかし,その経路の始発駅や終着駅を外方の駅として運賃経路を伸ばしたときに運賃が安くなる場合は,伸ばした駅を乗車券の始発駅や終着駅として安い運賃を採用する.
ここでは,分割することは考えずに入力された経路を内包し,最も安く乗車することができる1枚の乗車券が出力される.

この運賃計算方法により,1枚の乗車券と分割乗車券のいずれを考えたときでも,経路を外方へ伸ばしたときに運賃が安くなることは無い.
よって,運賃は運賃計算キロに対して単調非減少であることが示せる.

\section{分割乗車券プログラム}

本研究では,運賃が運賃計算キロに対して単調非減少であるという前提に基づき,探索対象を単純パスに限定する.
経路に閉路が含まれる場合,その閉路を取り除いた経路の総距離は必ず元の経路よりも短くなるか等しくなる.
運賃の単調非減少性より,閉路を含む経路が閉路を含まない経路よりも安くなることはない.
同様に,経路の一部を折り返すような経路については,折り返し地点で乗車券を分割する必要があるが,重複区間を取り除いた単純な経路と比較して運賃が安くなることはない.
したがって,最安解となる経路の候補は単純パスの中に必ず存在する.

また,探索アルゴリズムの選定において,A*アルゴリズムなどのヒューリスティック探索を用いることで計算量を削減する手法が考えられる.
しかし,JRの運賃制度には「営業キロが実キロ(物理的な距離)よりも短く設定されている区間」や「経路特定区間(遠回りの経路でも最短経路の距離で計算する特例)」が存在する.
このため,物理的な距離等をヒューリスティック関数 $h(n)$ として用いた場合,推定値が実際のコストを上回る(過大評価する)ケースが発生し,A*アルゴリズムの許容性条件 $h(n) \leqq h^*(n)$ を満たさなくなる.
これにより,最安経路の見逃しが発生するリスクがあるため,本研究ではヒューリスティックを用いない Dijkstra's Algorithm および Yen's Algorithm を採用し,解の正当性を保証する.

提案手法の手順は以下の通りである.

\begin{enumerate}
    \item \textbf{基準経路の探索:}\\
    Dijkstra's algorithmを用い,発駅から着駅までの運賃計算キロが最短となる経路を探索する.
    これを探索の基準経路とし,その距離を $d_{min}$ とする.

    \item \textbf{候補経路の列挙:}\\
    Yen's Algorithmを用い,第$K$最短経路を順次探索して候補経路リストに追加する.
    ここで,探索の打ち切り条件(距離の閾値 $d_{limit}$)は,第3章で述べた論理に加え,特定都区市内制度や外方延伸による営業キロと実キロの乖離を考慮し,以下の式で定義する.
    
    \[
    d_{limit} = d_{min} \times \frac{\text{基準経路のキロ単価}}{\text{最安キロ単価 (7.05)}} + \alpha
    \]
    
    ここで,$\alpha$ は特定都区市内制度や経路特定区間において想定される,外方へ伸ばす営業キロの最大値である.
    探索距離がこの $d_{limit}$ を超えた時点で,それ以上の探索を打ち切る.

    \item \textbf{各候補経路における最適分割の計算(特例の考慮):}\\
    列挙された各候補経路について,個別に最安の分割パターンを算出する.
    具体的には,経路上の各駅をノードとし,任意の2駅間の運賃をエッジの重みとする有向非巡回グラフ(DAG)を仮想的に構築し,その上での最短経路問題を解く.
    
    なお,エッジの重み(区間運賃)の決定においては,単純な営業キロ計算だけでなく,以下の特例適用時の運賃と比較し,最小値を採用する.
    \begin{itemize}
        \item \textbf{特定都区市内制度:} 発着駅が制度対象エリアにある場合,中心駅からの営業キロを用いた計算.
        \item \textbf{外方への延伸(前途放棄):} 対象区間を外方の駅まで延伸させることで,遠距離逓減制や特定運賃の適用により運賃が安価となる場合,その延伸先の運賃.
    \end{itemize}

    \item \textbf{解の出力:}\\
    全ての候補経路の中で最も運賃が安い分割パターンと経路を出力する.
\end{enumerate}

%%%%%%%%%%%%%%%%%%%%%
% 4章
%%%%%%%%%%%%%%%%%%%%%
\chapter{システムの実装} \label{chapter:implementation}

\section{開発環境と使用技術}
\begin{itemize}
    \item TypeScript(Node.js)
\end{itemize}

\section{運賃・経路データの構築}
以下のデータをそれぞれプログラムへ読み込む.
なお,営業キロと賃率換算キロ及び擬制キロはJR時刻表\cite{jikoku}のデータを利用している.

\begin{itemize}
    \item 入力用データリスト
    \begin{itemize}
        \item 駅名カナデータ,路線駅データ,乗換データ
    \end{itemize}
    \item 特定市内データ
    \begin{itemize}
        \item 名前(例:札幌市内),駅の集合
    \end{itemize}
    \item 経由印字データリスト
    \begin{itemize}
        \item 路線名に紐づく経由印字
    \end{itemize}
    \item セグメントデータリスト
    \begin{itemize}
        \item 営業キロ,換算キロ,旅客会社
    \end{itemize}
    \item 特定運賃リスト
    \item 経路の置き換えリスト
    \item 山手線内データリスト
\end{itemize}

\subsection{主要モジュールの実装}

\subsubsection{運賃計算プログラム}
分割乗車券プログラムから呼び出されるプログラムであり,発駅と着駅の他に経路も指定する必要がある.
入力として受け取った発駅から着駅までの経路を含む乗車券のうち最安のものを返す.
ここでは,分割することは考えずに1枚の乗車券として出力する.
ただし,このときに運賃計算経路を外方に伸ばした時に運賃が安くなる場合は乗車券の区間と運賃を伸ばした経路に置き換え,増運賃区間にまたがる場合や鉄道バリアフリー料金の10円も加えた運賃を返す.

\subsubsection{分割乗車券プログラム}
本節では,第3章で定義した探索アルゴリズムを,TypeScriptを用いて具体的なソフトウェアとして実装する際の詳細について述べる.
実装においては,計算速度とメモリ効率を重視し,以下のデータ構造および処理方式を採用した.

\begin{enumerate}
    \item \textbf{Priority Queueの実装:}\\
    基準経路探索およびYen's Algorithmの内部処理で用いるDijkstra法の実装においては,探索ノードの取り出しを高速化するためにBinary Heapを用いたPriority Queueを採用した.
    これにより,ノード数が膨大になる長距離区間の探索においても,計算量の増大を抑えている.

    \item \textbf{Yen's Algorithmにおける経路管理:}\\
    第$K$最短経路の探索において,候補となる経路群はメモリ上でリスト構造として管理される.
    第3章で定義した打ち切り閾値 $d_{limit}$ の判定は,経路が1本見つかるごとに動的に行われ,条件を満たさなくなった時点で直ちにループを脱出する設計とした.
    これにより,無駄な探索処理が走ることを防いでいる.
\end{enumerate}

%%%%%%%%%%%%%%%%%%%%%
% 5章
%%%%%%%%%%%%%%%%%%%%%
\chapter{評価実験と考察} \label{chapter:evaluation}

\section{実験概要}
評価に用いた計算機のスペック:MacBook Pro, Apple M4, 16GB

\section{計算結果の正当性評価}

\section{性能評価}
計算時間を計測する.

\section{考察}
計算時間がかかったケースがあれば,その原因を考える.

%%%%%%%%%%%%%%%%%%%%%
% 6章
%%%%%%%%%%%%%%%%%%%%%
\chapter{結論} \label{chapter:conclusion}
\section{本研究の総括}
本研究では,JR分割乗車券の最適解を探索するプログラムをDijkstra's algorithmを用いて開発した.

\section{将来の展望と今後の課題}
\begin{itemize}
    \item 展望:新幹線や特急料金を含めた最適化,Webサービスとしての一般公開,モバイルアプリ化など.
    \item 課題:運賃改定への自動追従システムの構築,計算速度の更なる向上,各種割引への対応.
\end{itemize}

%%%%%%%%%%%%%%%%%%%%%
% 謝辞
%%%%%%%%%%%%%%%%%%%%%
\syaji
\par
本論文について貴重な御助言を頂いた指導教員や研究室のメンバー方には心より感謝致します.

%%%%%%%%%%%%%%%%%%%%%
% 参考文献
%%%%%%%%%%%%%%%%%%%%%
\begin{thebibliography}{99}
\addcontentsline{toc}{chapter}{参考文献}

\bibitem{dijkstra}
E.W. Dijkstra, A note on two problems in connexion with graphs, Numerische Mathematik, vol.~1, pp.~269--271, 1959.

\bibitem{yen}
Jin Y. Yen, Finding the K Shortest Loopless Paths in a Network, Management Science, vol.~17, no.~11, pp.~712--716, 1971.

\bibitem{oba}
oba,乗車券分割プログラム,入手先〈\url{https://bunkatsu.jp/}〉(参照2025-12-25).

\bibitem{hirata}
平田 直也, 中桐 斉之, スマートフォンに特化した乗車券分割アプリの開発, 第78回全国大会講演論文集, vol.~2016, no.~1, pp.~387--388, 2016.

\bibitem{ryokaku}
JR東日本,旅客営業規則,入手先〈\url{https://www.jreast.co.jp/ryokaku/}〉(参照2025-12-25).

\bibitem{jikoku}
時刻表編集部,JR時刻表 2025年10月号,交通新聞社,2025.

\bibitem{trrc}
川浦 龍一,遠い方が安くなる?! なぜなにJR旅客営業制度 --きっぷのふしぎ--,交通法規研究会,東京,2024.

\end{thebibliography}

%%%%%%%%%%%%%%%%%%%%%
% 付録
%%%%%%%%%%%%%%%%%%%%%
\appendix
\def\labelenumi{(\theenumi)}
\chapter{運賃計算方法と特例}

片道乗車券の運賃計算する際の経路補正に関する規則のみを抜粋して掲載する.規則の全文については参考文献\cite{ryokaku}を参照されたい.
\\\\
\noindent\textbf{(営業キロ)}\\
\noindent\textbf{第14条}\\
旅客運賃・料金の計算その他の旅客運送の条件をキロメートルをもって定める場合は、別に定める場合を除き、営業キロによる。\\

\noindent\textbf{2}\\
前条の営業キロは、旅客の乗車する発着区間に対する駅間のキロ数による。\\

\noindent\textbf{(運賃計算キロ)}\\
\noindent\textbf{第14条の2}\\
前条の規定によるほか、幹線と地方交通線を連続して乗車する場合(幹線と地方交通線の中間に当社と通過連絡運輸を行う鉄道・軌道・航路又は自動車線が介在する場合で、これらを通じて連続乗車するときを含む。以下同じ。)の旅客運賃を計算するときは、旅客の乗車する発着区間のうち、地方交通線の乗車区間に対する営業キロを賃率比に応じて換算したもの(以下、北海道旅客鉄道株式会社、東日本旅客鉄道株式会社、東海旅客鉄道株式会社及び西日本旅客鉄道株式会社にあっては「賃率換算キロ」、四国旅客鉄道株式会社及び九州旅客鉄道株式会社にあっては「擬制キロ」という。)と幹線の乗車区間に対する営業キロを合算したもの(以下「運賃計算キロ」という。)による。\\

\noindent\textbf{(擬制キロ)}\\
\noindent\textbf{第14条の3}\\
第14条の規定にかかわらず、四国旅客鉄道会社線又は九州旅客鉄道会社線の地方交通線内各駅相互間を乗車する場合の旅客運賃を計算するときは、前条第1項に定める擬制キロによる。\\

\noindent\textbf{(他の旅客鉄道会社線を通じて連続乗車する場合の営業キロ、賃率換算キロ、擬制キロ又は運賃計算キロの通算)}\\
\noindent\textbf{第14条の4}\\
当社線と他の旅客鉄道会社線を通じて連続乗車する場合の営業キロ、賃率換算キロ、擬制キロ又は運賃計算キロは、旅客の乗車区間に対し、第14条又は第14条の2の規定を適用して計算したものによる。\\

\noindent\textbf{2}\\
前項の規定による営業キロ、賃率換算キロ、擬制キロ又は運賃計算キロは、旅客運賃・料金の計算その他この規則に定める取扱いをする場合に適用する。\\

\noindent\textbf{(普通乗車券の発売方)}\\
\noindent\textbf{第26条の2}\\
次の各号に掲げる場合は、前条及び第68条第4項の規定により、それぞれ片道乗車券又は連続乗車券を発売する。
\begin{enumerate}
    \item 環状線一周となる経路の場合は、片道乗車券を発売する。
\end{enumerate}

\noindent\textbf{(旅客運賃・料金計算上の経路等)}\\
\noindent\textbf{第67条}\\
旅客運賃・料金は、旅客の実際乗車する経路及び発着の順序によって計算する。

\noindent\textbf{(旅客運賃・料金計算上の営業キロ等の計算方)}\\
\noindent\textbf{第68条}\\
営業キロ又は擬制キロを使用して旅客運賃を計算する場合は、別に定める場合を除いて、次の各号により営業キロ又は擬制キロを通算して計算する。
\begin{enumerate}
    \item 営業キロ又は擬制キロは、同ー方向に連続する場合に限り、これを通算する。
\end{enumerate}

\noindent\textbf{2}\\
前項の規定は、運賃計算キロを使用して幹線と地方交通線を連続して乗車するときの旅客運賃を計算する場合に準用する。\\

\noindent\textbf{4}\\
前各項の規定により、旅客運賃・料金を計算する場合で次の各号の1に該当するときは、当該各号に定めるところによって計算する。
\begin{enumerate}
    \item 計算経路が環状線1周となる場合は、環状線1周となる駅の前後の区間の営業キロ、擬制キロ又は運賃計算キロを打ち切って計算する。
    \item 計算経路の一部若しくは全部が復乗となる場合は、折返しとなる駅の前後の区間の営業キロ、擬制キロ又は運賃計算キロを打ち切って計算する。
\end{enumerate}

\noindent\textbf{(特定区間における旅客運賃・料金計算の営業キロ又は運賃計算キロ)}\\
\noindent\textbf{第69条}\\
第67条の規定にかかわらず、次の各号に掲げる区間の普通旅客運賃・料金は、その旅客運賃・料金計算経路が当該各号末尾のかっこ内の両線路にまたがる場合を除いて、○印の経路の営業キロ(第9号については運賃計算キロ。ただし、岩国・櫛ヶ浜間相互発着の場合にあっては営業キロ)によって計算する。この場合、各号の区間内については、経路の指定を行わない。

\begin{enumerate}
    \item 大沼以遠(仁山方面)の各駅と、森以遠(石倉方面)の各駅との相互間\\
        \qquad    
        $
            \Bigg(
                \begin{tabular}{l}
                     東森駅経由函館本線\\
                    ○大沼公園駅経由函館本線
                \end{tabular}
            \Bigg)
        $
    
    \item 日暮里以遠(鶯谷又は三河島方面)の各駅と、赤羽以遠(川口、北赤羽又は十条方面)の各駅との相互間\\
        \qquad    
        $
            \Bigg(
                \begin{tabular}{l}
                     尾久経由東北本線\\
                    ○王子経由東北本線
                \end{tabular}
            \Bigg)
        $

    \item 赤羽以遠(尾久、東十条又は十条方面)の各駅と、大宮以遠(土呂、宮原又は日進方面)の各駅との相互間\\
        \qquad    
        $
            \Bigg(
                \begin{tabular}{l}
                     戸田公園・与野本町経由東北本線\\
                    ○川口・浦和経由東北本線
                \end{tabular}
            \Bigg)
        $

    \item 品川以遠(高輪ゲートウェイ又は大崎方面)の各駅と、鶴見以遠(新子安、国道又は羽沢横浜国大方面)の各駅との相互間\\
        \qquad    
        $
            \Bigg(
                \begin{tabular}{l}
                     西大井経由東海道本線\\
                    ○大井町経由東海道本線
                \end{tabular}
            \Bigg)
        $

    \item 東京以遠(有楽町又は神田方面)の各駅と、蘇我以遠(鎌取又は浜野方面)の各駅との相互間\\
        \qquad    
        $
            \Bigg(
                \begin{tabular}{l}
                     京葉線\\
                    ○総武本線・外房線
                \end{tabular}
            \Bigg)
        $

    \item 山科以遠(京都方面)の各駅と、近江塩津以遠(新疋田方面)の各駅との相互間\\
        \qquad    
        $
            \Bigg(
                \begin{tabular}{l}
                     東海道本線・北陸本線\\
                    ○湖西線
                \end{tabular}
            \Bigg)
        $

    \item 大阪以遠(塚本又は新大阪方面)の各駅と、天王寺以遠(東部市場前又は美章園方面)の各駅との相互間\\
        \qquad    
        $
            \Bigg(
                \begin{tabular}{l}
                     福島経由大阪環状線\\
                    ○天満経由大阪環状線
                \end{tabular}
            \Bigg)
        $

    \item 三原以遠(糸崎方面)の各駅と、海田市以遠(向洋方面)の各駅との相互間\\
        \qquad    
        $
            \Bigg(
                \begin{tabular}{l}
                     呉線\\
                    ○山陽本線
                \end{tabular}
            \Bigg)
        $

    \item 岩国以遠(和木方面)の各駅と、櫛ヶ浜以遠(徳山方面)の各駅との相互間\\
        \qquad    
        $
            \Bigg(
                \begin{tabular}{l}
                     山陽本線\\
                    ○岩徳線
                \end{tabular}
            \Bigg)
        $

\end{enumerate}

\noindent\textbf{第70条}\\
第67条の規定にかかわらず、旅客が次に掲げる図の太線区間を通過する場合の普通旅客運賃・料金は太線区間内の最も短い営業キロによって計算する。この場合、太線内は、経路の指定を行わない。\\
\includegraphics[scale=0.9]{70.png}

\noindent\textbf{2}\\
蘇我以遠(鎌取又は浜野方面)の各駅と前条第1項第5号に掲げるいずれかの経路を経由して前項に掲げる図の太線区間を大久保以遠(東中野方面)、三河島以遠(南千住方面)、川口以遠(西川口方面)又は北赤羽以遠(浮間舟渡方面)へ通過する場合の普通旅客運賃・料金は、第67条及び前条第1項第5号の規定にかかわらず、外房線蘇我・千葉間、総武本線千葉・錦糸町間及び前項に掲げる図の太線区間内の最も短い経路の営業キロによって計算する。

\noindent\textbf{(鉄道駅バリアフリー料金)}\\
\noindent\textbf{第140条}\\
次の各号に掲げる区間内相互発着となる区間に乗車する場合は、鉄道駅バリアフリー料金を収受する。
\begin{enumerate}
    \item 第78条第2項第1号に定める東京附近における電車特定区間及び第80条の規定を適用する区間(同条第1項第1号から第4号の区間にかかるものに限る。)
    \item 第78条第2項第2号に定める大阪附近における電車特定区間及び第80条の規定を適用する区間(同条第1項第5号から第14号及び同条第2項の区間にかかるものに限る。)
    \item 東海道本線(新幹線)中豊橋・岐阜羽島間、東海道本線中豊橋・大垣間、武豊線、中央本線中多治見・名古屋間、関西本線中名古屋・四日市間(ただし、対象区間のみを経由して乗車する場合に限る。)
\end{enumerate}

\noindent\textbf{2}\\
前項の規定により収受する鉄道駅バリアフリー料金は、次の各号に定めるとおりとする。
\begin{enumerate}
    \item 前項第1号に掲げる区間内相互発着となる区間に乗車する場合\\
    イ 大人片道普通旅客運賃とあわせ収受する額\\
      片道乗車あたり10円
    \item 前項第2号及び第3号に掲げる区間内相互発着となる区間に乗車する場合\\
    イ 大人片道普通旅客運賃とあわせ収受する額\\
      片道乗車あたり10円
\end{enumerate}

\end{document}