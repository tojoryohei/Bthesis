\documentclass[a4j,12pt]{jreport}
%\documentclass{jreport}
\usepackage[dvipdfmx]{graphicx}
\usepackage{amsmath,amssymb}
\usepackage{here}
\usepackage{algorithm}
\usepackage{algpseudocode}
\usepackage{hhline}
\usepackage[hang,small,bf]{caption}
\usepackage[subrefformat=parens]{subcaption}
\usepackage{url}
\captionsetup{compatibility=false}

\def\syaji{ \chapter*{謝辞} \addcontentsline{toc}{chapter}{謝辞}}
\renewcommand{\bibname}{参考文献}
\setlength{\textheight}{\paperheight}
\setlength{\topmargin}{4.6mm}
\addtolength{\topmargin}{-\headheight}
\addtolength{\topmargin}{-\headsep}
\addtolength{\topmargin}{-\headheight}
\addtolength{\textheight}{-60mm}

\setlength{\textwidth}{\paperwidth}
\setlength{\oddsidemargin}{-0.4mm}
\setlength{\evensidemargin}{-0.4mm}
\addtolength{\textwidth}{-50mm}

\begin{document}

%%%%%%%%%%%%%%%%%%%%%
% 表紙
%%%%%%%%%%%%%%%%%%%%%
\thispagestyle{empty}
\begin{center}
  \begin{Large}
    \vspace*{0.7cm}
    {\large 卒業研究論文}\\
    \vspace*{2.5cm}
    {\LARGE\bf JR最安分割乗車券探索システムの開発}\\
    \vspace*{7.5cm}
    東條 涼平\\
    学籍番号\hspace*{1zw}22D8102015D\\
    \vspace*{2.5cm}
    中央大学理工学部情報工学科\hspace*{1zw} アルゴリズム理論基礎研究室\\
    \vspace*{3.0cm}
    2026年3月\\
  \end{Large}
\end{center}


%%%%%%%%%%%%%%%%%%%%%
% 概要
%%%%%%%%%%%%%%%%%%%%%
\newpage
\renewcommand{\baselinestretch}{1.25} \selectfont
\pagenumbering{roman}


\begin{center} {\large \bf{概 要}} \end{center}

JR線を利用する際,安く移動をしようとするとき,通しで購入する乗車券の運賃よりも乗車区間を途中で区切って複数枚の乗車券を購入するほうが安い場合がある.
この区間を分けて購入した乗車券を分割乗車券という.
従来,任意の駅間において分割乗車券を考慮したときに最安解を出すためには,最適解のルートとして考えられる経路の全てで分割乗車券の運賃を比較する必要があったが,このとき分割乗車券の経路が最短経路でない場合もあるため,複数の経路を試す必要があった.
そこで,Dijkstra's algorithm\cite{dijkstra}とYen's Algorithm\cite{yen}を使って最安となりうる経路の候補を出し,それぞれに対して分割経路を計算をして,それらの結果をもとに最安解を出力することにより,経路を指定せずに発駅と着駅の入力のみで最適解を出すプログラムを作成した.

\vspace{1zw} \noindent
{\bf キーワード: }分割乗車券,経路探索,JR運賃計算,グラフ理論

%%%%%%%%%%%%%%%%%%%%%
% 目次
%%%%%%%%%%%%%%%%%%%%%
\tableofcontents


\newpage
\pagenumbering{arabic}

%%%%%%%%%%%%%%%%%%%%%
% 1章
%%%%%%%%%%%%%%%%%%%%%
\chapter{序論} \label{chapter:intro}

\section{研究の背景}
JRでは営業キロを一定の幅に区切って運賃を設定しているため,乗車券の区間を分割して購入したほうが安くなる場合がある.
区間を分けて乗車券を購入することを分割乗車券といい,2000年以前からインターネット上で知られていた.
発駅と着駅のほかに経路を指定した場合であれば,最安解を出力する分割乗車券プログラム\cite{oba}が既に存在する.
多くの場合では,発駅から着駅までを結ぶ最短経路上に存在するが,都心のような路線が狭い範囲で路線が密集している場合など,最短経路上に分割駅が無い場合もある.
そのため,旅客が最安解である経路を特定することは困難である.

\section{研究の対象}
今回の研究では,大人の片道乗車券を計算の対象とする.
また,新幹線を使った経路では乗車券の他に特急券が必ず必要となるため今回の研究では在来線のみを探索の対象とする.
片道乗車券の運賃計算には多くの例外があり複雑である.
例えば,同一の区間であっても複数の正規運賃が存在することもある.
この研究で扱う乗車券の運賃と経路は,運賃計算の手順が記載されている旅客営業規則に則って計算をして,駅の窓口や自動券売機で実際に購入できるものを扱うこととする.

\section{関連研究・従来の手法}
分割乗車券プログラムをWebアプリとして公開しているサイトが存在する.このサイトでは,発駅と着駅の他に経路を指定する必要があり,その経路上で最安分割乗車券を探索する.
また,鉄道運賃の計算に関する研究として,森田ら\cite{morita}は,運賃計算ルールをグラフ構造に反映させたFarenetを提案し,高速な最安経路探索を実現している.
さらに,分割乗車券に特化した先行研究として,平田ら\cite{hirata}によるスマートフォンに特化した乗車券分割アプリの事例がある.
しかし,平田らの手法では最短経路上で分割回数が1回に限定されているという制約があり,探索エリアも関西の一部エリアのみである.

\section{研究の目的と意義}
旅客らに実用的な節約手段を提供でき,複雑な制約を持つ大規模グラフ問題に対する効率的な解法を高速な計算時間で実装する.

%%%%%%%%%%%%%%%%%%%%%
% 2章
%%%%%%%%%%%%%%%%%%%%%
\chapter{JR運賃制度と問題の定義} \label{chapter:fare_system}

\section{JR運賃計算の基礎}
JRは6つの旅客鉄道会社と1つの貨物鉄道会社から構成される.
この研究では旅客鉄道会社のみを扱い,これ以降は通称を使う.
表\ref{tab:jr_companies}に各社の通称を示す.

\begin{table}[H]
\centering
\caption{旅客鉄道会社一覧}
\label{tab:jr_companies}
\begin{tabular}{|c|c|} \hline
会社名 & 通称 \\ \hline
北海道旅客鉄道 & JR北海道 \\ \hline
東日本旅客鉄道 & JR東日本 \\ \hline
東海旅客鉄道 & JR東海 \\ \hline
西日本旅客鉄道 & JR西日本 \\ \hline
四国旅客鉄道 & JR四国 \\ \hline
九州旅客鉄道 & JR九州 \\ \hline
\end{tabular}
\end{table}

各旅客鉄道会社には旅客営業規則が定められているが,それぞれに内容の差異は無い.
ここでは,JR東日本の旅客営業規則\cite{ryokaku}を引用する.

旅客営業規則では\textbf{乗車券}は以下のとおりに定められている.
\begin{itemize}
    \item 普通乗車券
    \begin{itemize}
        \item 片道乗車券
        \item 往復乗車券
        \item 連続乗車券
    \end{itemize}
    \item 定期乗車券
    \begin{itemize}
        \item 通勤定期乗車券
        \item 通学定期乗車券
        \item 特殊定期乗車券
        \begin{itemize}
            \item 特別車両定期乗車券
            \item 特殊均一定期乗車券
        \end{itemize}
    \end{itemize}
    \item 普通回数乗車券
    \item 団体乗車券
    \item 貸切乗車券
\end{itemize}

今回の研究では,普通乗車券のうち片道乗車券のみを扱う.

JRの運賃計算で使われる駅間はそれぞれ以下の情報を持つ.
\begin{itemize}
    \item 「幹線」か「地方交通線」のどちらか
    \item 営業キロ
    \item 擬制キロ(地方交通線のみ)
    \item 所属する旅客会社
\end{itemize}
原則として,これらの情報をもとに運賃計算がなされる.

このときに,片道乗車券における以下の2つのルールを守らなくてはならない.
\begin{itemize}
    \item 環状線を超える経路であってはならない.
    \item 経路が重複してはならない.
\end{itemize}

擬制キロとは,駅間が地方交通線だった場合に設定される.
これは,運賃を計算する上で幹線と地方交通線の間で賃率が違うことから,営業キロにおおよそ1.1倍をした運賃計算用の値である.
JR各社によって,「換算キロ」や「擬制キロ」,「運賃計算キロ」のように呼び方が異なるが全て同じものを表し,これ以降は「擬制キロ」に統一することとする.
また,一部区間を相互発着となる区間に乗車する場合は,運賃の他に\textbf{鉄道駅バリアフリー料金}が収受される.
今回の研究では,運賃に鉄道駅バリアフリー料金をあわせた金額を運賃とする.

\section{問題の定式化}
分割乗車券プログラムを開発するにあたって,運賃計算プログラムを作る必要がある.
ICカード運賃のほうが安い場合もあるが,今回の研究では考慮しない.
この運賃計算プログラムでは,与えられた経路の運賃を出力するだけでなく,入力された経路の始発駅や終着駅を外方の駅としたときに運賃が安くなる場合,その駅を乗車券の始発駅や終着駅とする.
その区間の一部を使いその乗車券の運賃を入力区間の運賃とする.
これを内方乗車と呼ぶ.

%%%%%%%%%%%%%%%%%%%%%
% 3章
%%%%%%%%%%%%%%%%%%%%%
\chapter{探索手法} \label{chapter:algorithm}

\section{最安分割乗車券の候補となる経路の探索アルゴリズム}
運賃計算経路を延長したときに運賃が安くなることはないから,閉路が含まれる経路が最安解だった場合,その閉路を取り除いた経路も最適解である.
同様にして,途中で折り返すような経路が最安解だった場合,重複区間を取り除いた経路がその閉路を取り除いた経路が高くなることはない.
したがって,最安解のうち経路が単純パスであり,経路単純に同じ駅が2回出てこなければ良い.

Dijkstra's algorithmにより,発駅から着駅までの最短擬制キロとなる経路を一つ選ぶ.
Yen's Algorithmを使って候補として分割乗車券の計算を行う.

その中で最安となる分割乗車券を全て出力する.

\section{運賃計算プログラム}
分割乗車券プログラムから呼び出されるプログラムであり,入力では発駅と着駅の他に経路も指定する必要がある.
入力された発駅から着駅までの経路を含む乗車券のうち最安のものを返す.
ここでは,分割することは考えずに1枚の乗車券として出力する.

運賃計算アルゴリズムは以下の通りである.
\begin{itemize}
    \item 特定運賃区間と一致する場合はその運賃を返す.
    \item セグメントデータを旅客会社ごとに分けて,まずはすべての区間をJR東日本・JR東海・JR西日本(以降これらを本州三社と呼ぶ)の運賃体系で計算をする.
    北海道・四国・九州ごとの区間でその会社ごとの運賃から本州三社の運賃の差を加算運賃として加える.
    また,増運賃区間にまたがる場合や鉄道バリアフリー料金も加えた運賃を返す.
\end{itemize}
ただし,このときに運賃計算経路を外方に伸ばした時に運賃が安くなる場合は乗車券の区間と運賃を伸ばした経路に置き換える.
出力された乗車券のうち,入力の経路のみを利用して,外方に伸ばした部分は乗車せずに権利放棄をする.

\section{分割乗車券プログラム}
\begin{enumerate}
    \item Dijkstra's algorithmを使い,入力された発駅と着駅の間で擬制キロが最短となるような経路を1つ選ぶ.
    \item (その経路のキロ単価)÷(最安のキロ単価)倍までの経路をYen's Algorithmを使って候補として分割乗車券の計算を行う.
    \item その中で最安となる分割乗車券を全て出力する.
\end{enumerate}

%%%%%%%%%%%%%%%%%%%%%
% 4章
%%%%%%%%%%%%%%%%%%%%%
\chapter{システムの実装} \label{chapter:implementation}

\section{開発環境と使用技術}
\begin{itemize}
    \item TypeScript(Node.js)
\end{itemize}

\section{運賃・経路データの構築}
以下のデータをそれぞれプログラムへ読み込む.
なお,営業キロと擬制キロはJR時刻表\cite{jikoku}のデータを利用している.

\begin{itemize}
    \item 入力用データリスト
    \begin{itemize}
        \item 駅名カナデータ,路線駅データ,乗換データ
    \end{itemize}
    \item 特定市内データ
    \begin{itemize}
        \item 名前(例:札幌市内),駅の集合
    \end{itemize}
    \item 経由印字データリスト
    \begin{itemize}
        \item 路線名に紐づく経由印字
    \end{itemize}
    \item セグメントデータリスト
    \begin{itemize}
        \item 営業キロ,換算キロ,旅客会社
    \end{itemize}
    \item 特定運賃リスト
    \item 経路の置き換えリスト
    \item 山手線内データリスト
\end{itemize}

\section{主要モジュールの実装}

\subsection{運賃計算プログラム}
分割乗車券プログラムから呼び出されるプログラムであり,発駅と着駅の他に経路も指定する必要がある.
入力として受け取った発駅から着駅までの経路を含む乗車券のうち最安のものを返す.
ここでは,分割することは考えずに1枚の乗車券として出力する.
アルゴリズムは以下の通りである.
\begin{itemize}
    \item 特定運賃区間と一致する場合はその運賃を返す.
    \item セグメントデータを旅客会社ごとに分けて,まずはすべての区間をJR東日本・JR東海・JR西日本(以降これらを本州三社と呼ぶ)の運賃体系で計算をする.
    北海道・四国・九州ごとの区間でその会社ごとの運賃から本州三社の運賃の差を加算運賃として加える.
    また,増運賃区間にまたがる場合や鉄道バリアフリー料金も加えた運賃を返す.
    ただし,このときに運賃計算経路を外方に伸ばした時に運賃が安くなる場合は乗車券の区間と運賃を伸ばした経路に置き換える.
\end{itemize}

\subsection{分割乗車券プログラム}
\begin{enumerate}
    \item Dijkstra's algorithmを使い,入力された発駅と着駅の間で擬制キロが最短となるような経路を1つ選ぶ.
    \item (その経路のキロ単価)÷(最安のキロ単価)倍までの経路をYen's Algorithmを使って候補として分割乗車券の計算を行う.
    \item 2で選んだ駅から隣駅までの(通しの)運賃,及び,その隣駅から発駅までの各駅における始発駅からの運賃とその隣駅までの運賃の和の中で最安の経路と運賃を計算する.
    
    運賃計算経路を延長したときに運賃が安くなることはないから,閉路が含まれる経路が最安解だった場合,その閉路を取り除いた経路も最適解である.
    同様にして,途中で折り返すような経路が最安解だった場合,重複区間を取り除いた経路がその閉路を取り除いた経路が高くなることはない.
    したがって,探索するべき経路は単純パスであり,
    よって単純に同じ駅が2回出てこなければ良い.

    \item 2で選んだ駅が着駅であったら計算結果を返して終了する.そうでなければ計算結果をキューに入れて2へ戻る.
\end{enumerate}

%%%%%%%%%%%%%%%%%%%%%
% 5章
%%%%%%%%%%%%%%%%%%%%%
\chapter{評価実験と考察} \label{chapter:evaluation}

本章では,実装したプログラムが目的を達成できたか(正しく,速く計算できるか)を評価する.

\section{実験概要}
評価に用いた計算機のスペック,テストケース(例:短距離,長距離,特定区間を含むなど)の選定基準について述べる.

\section{計算結果の正当性評価}
プログラムが算出した解が,本当に「最適(最安)」であるかを検証する.
(例:既知の分割パターンや,他の信頼できるサービスとの結果比較.)

\section{性能評価(計算速度)}
様々なテストケースにおいて,最適解を算出するまでに要した時間を計測し,結果を表やグラフで示す.
(例:Dijkstra's algorithm(またはヒューリスティックなし)の場合との計算時間比較.)

\section{考察}
実験結果から何が言えるかを考察する.
(例:設計したヒューリスティック関数はどの程度有効であったか.計算時間がかかったケースがあれば,その原因は何か.)

%%%%%%%%%%%%%%%%%%%%%
% 6章
%%%%%%%%%%%%%%%%%%%%%
\chapter{結論} \label{chapter:conclusion}

本章では,研究全体の成果を総括し,今後の展望を述べる.

\section{本研究の総括}
本研究では,JR分割乗車券の最適解を探索するプログラムをDijkstra's algorithmを用いて開発した.

\section{将来の展望と今後の課題}
\begin{itemize}
    \item 展望:新幹線や特急料金を含めた最適化,Webサービスとしての一般公開,モバイルアプリ化など.
    \item 課題:運賃改定への自動追従システムの構築,計算速度の更なる向上,各種割引への対応.
\end{itemize}

%%%%%%%%%%%%%%%%%%%%%
% 謝辞
%%%%%%%%%%%%%%%%%%%%%
\syaji
\par
本論文について貴重な御助言を頂いた指導教員や研究室のメンバー方には心より感謝致します.

%%%%%%%%%%%%%%%%%%%%%
% 参考文献
%%%%%%%%%%%%%%%%%%%%%
\begin{thebibliography}{99}
\addcontentsline{toc}{chapter}{参考文献}

\bibitem{dijkstra}
E.W. Dijkstra, A note on two problems in connexion with graphs, Numerische Mathematik, vol.~1, pp.~269--271, 1959.

\bibitem{yen}
Jin Y. Yen, Finding the K Shortest Loopless Paths in a Network, Management Science, vol.~17, no.~11, pp.~712--716, 1971.

\bibitem{oba}
oba,乗車券分割プログラム,入手先〈\url{https://bunkatsu.jp/}〉(参照2025-12-25).

\bibitem{ryokaku}
JR東日本,旅客営業規則,入手先〈\url{https://www.jreast.co.jp/ryokaku/}〉(参照2025-12-25).

\bibitem{morita}
森田 隼史, 池上 敦子, 菊地 丞, 山口 拓真, 中山 利宏, 大倉 元宏, 鉄道運賃計算アルゴリズム : Suica/PASMO利用可能範囲のJR東日本510駅の運賃を対象とした場合, 日本オペレーションズ・リサーチ学会和文論文誌, vol.~54, pp.~1--22, 2011.

\bibitem{hirata}
平田 直也, 中桐 斉之, スマートフォンに特化した乗車券分割アプリの開発, 第78回全国大会講演論文集, vol.~2016, no.~1, pp.~387--388, 2016.

\bibitem{jikoku}
JR時刻表 2025年10月号.

\end{thebibliography}

%%%%%%%%%%%%%%%%%%%%%
% 付録
%%%%%%%%%%%%%%%%%%%%%
\appendix
\chapter{付録}

\section{複雑な運賃計算規則の詳細リスト}
本文中では概略に留めた,運賃計算の例外ルール(例:特定区間の詳細)などをまとめる.

\section{ヒューリスティック関数の許容性に関する数学的証明}
(もしあれば)

\section{実装したプログラムの主要なコード片}
(もしあれば)

\end{document}