\documentclass[a4j,12pt]{jreport}
%\documentclass{jreport}
\usepackage[dvipdfmx]{graphicx}
\usepackage{booktabs}
\usepackage{amsmath,amssymb}
\usepackage{here}
\usepackage{algorithm}
\usepackage{algpseudocode}
\usepackage{hhline}
\usepackage{makecell}
\usepackage{tabularx}
\usepackage[hang,small,bf]{caption}
\usepackage[subrefformat=parens]{subcaption}
\usepackage{url}
\usepackage{tikz}
\captionsetup{compatibility=false}
\usetikzlibrary{arrows.meta, positioning, calc}

\def\syaji{ \chapter*{謝辞} \addcontentsline{toc}{chapter}{謝辞}}
\renewcommand{\bibname}{参考文献}
\setlength{\textheight}{\paperheight}
\setlength{\topmargin}{4.6mm}
\addtolength{\topmargin}{-\headheight}
\addtolength{\topmargin}{-\headsep}
\addtolength{\topmargin}{-\headheight}
\addtolength{\textheight}{-60mm}

\setlength{\textwidth}{\paperwidth}
\setlength{\oddsidemargin}{-0.4mm}
\setlength{\evensidemargin}{-0.4mm}
\addtolength{\textwidth}{-50mm}

\begin{document}

%%%%%%%%%%%%%%%%%%%%%
% 表紙
%%%%%%%%%%%%%%%%%%%%%
\thispagestyle{empty}
\begin{center}
  \begin{Large}
    \vspace*{0.7cm}
    {\large 卒業研究論文}\\
    \vspace*{2.5cm}
    {\LARGE\bf JR最安分割乗車券探索システムの開発}\\
    \vspace*{7.5cm}
    東條 涼平\\
    学籍番号\hspace*{1zw}22D8102015D\\
    \vspace*{2.5cm}
    中央大学理工学部情報工学科\hspace*{1zw} アルゴリズム理論基礎研究室\\
    \vspace*{3.0cm}
    2026年3月\\
  \end{Large}
\end{center}


%%%%%%%%%%%%%%%%%%%%%
% 概要
%%%%%%%%%%%%%%%%%%%%%
\newpage
\renewcommand{\baselinestretch}{1.25} \selectfont
\pagenumbering{roman}


\begin{center} {\large \bf{概 要}} \end{center}

JR線を利用する際,安く移動をしようとするとき,通しで購入する乗車券の運賃よりも乗車区間を途中で区切って複数枚の乗車券を購入するほうが安い場合がある.
この区間を分けて購入した乗車券を分割乗車券という.
最安の分割乗車券の経路が最短経路でない場合もあるため,従来の手法では,任意の駅間において分割乗車券を考慮したときに最安解を出すためには,最適解のルートとして考えられる経路の全てで分割乗車券の運賃を比較する必要があった.
そこで,Dijkstra's algorithm\cite{dijkstra}とYen's Algorithm\cite{yen}を使って最安となりうる経路の候補を出し,それぞれに対して分割経路を計算をして,それらの結果をもとに最安解を出力することにより,経路を指定せずに発駅と着駅の入力のみで最適解を出すプログラムを作成した.

\vspace{1zw} \noindent
{\bf キーワード: }分割乗車券,経路探索,JR運賃計算,グラフ理論

%%%%%%%%%%%%%%%%%%%%%
% 目次
%%%%%%%%%%%%%%%%%%%%%
\tableofcontents


\newpage
\pagenumbering{arabic}

%%%%%%%%%%%%%%%%%%%%%
% 1章
%%%%%%%%%%%%%%%%%%%%%
\chapter{序論} \label{chapter:intro}

JRでは営業キロを一定の幅に区切って運賃を設定しているため,乗車券の区間を分割して購入したほうが安くなる場合がある.
区間を分けて乗車券を購入することを分割乗車券といい,2000年以前からインターネット上で知られていた.
発駅と着駅のほかに経路を指定した場合であれば,最安解を出力する分割乗車券プログラム\cite{oba}が既に存在する.
多くの場合では,発駅から着駅までを結ぶ最短経路上に分割駅が存在するが,都心のような路線が狭い範囲で路線が密集している場合など,最短経路上に分割駅が無い場合もある.
そのため,旅客が最安解である経路を特定することは困難である.

分割乗車券プログラムをWebアプリとして公開しているサイト\cite{oba}では,発駅と着駅の他に経路を指定する必要があり,その経路上で最安分割乗車券を探索する.
また,分割乗車券に特化した先行研究として,平田ら\cite{hirata}によるスマートフォンに特化した乗車券分割アプリの事例がある.
しかし,平田らの手法では最短経路上で分割回数が1回に限定されているという制約があり,探索エリアも関西の一部エリアのみである.
既存のWebサービスにおいても,利用者が経路を事前に指定する必要があるため,迂回路を含んだ真に最安となる経路を利用者自身が見つけ出すことは困難である.
したがって,経路指定を必要とせず,かつ分割回数やエリアに制限のない探索手法が求められている.

本研究の目的は,発駅と着駅のみを入力とし,分割乗車券を含む最安経路を自動的に導出するシステムを構築することである.
本研究の主な貢献は以下の3点である.
\begin{enumerate}
    \item \textbf{経路指定不要の探索:}\\
        Dijkstra's Algorithm と Yen's Algorithmを組み合わせることで,利用者が経路を指定することなく,分割乗車券によって安価となる迂回経路も含めた探索を可能にした.
    \item \textbf{複雑な運賃規則の網羅:}\\
        「経路特定区間」や「遠距離逓減に伴う逆転現象」など,単純なグラフ探索では扱えないJR特有の運賃規則をアルゴリズムに組み込み,実用的な解の正当性を向上させた.
    \item \textbf{厳密解探索における計算コストと課題の明確化:}\\
        探索打ち切り条件の最適化や,必要な区間運賃のみを計算する動的な重み付けなどの高速化手法を実装し,その挙動を長距離区間において検証した.
        評価の結果,現状のアルゴリズムではWebサービス水準の即応性確保が困難となる計算量のリスクを定量的に示し,複雑な運賃制度を厳密に考慮した探索におけるボトルネックを特定した.
\end{enumerate}

%%%%%%%%%%%%%%%%%%%%%
% 2章
%%%%%%%%%%%%%%%%%%%%%
\chapter{JR運賃制度と問題の定義} \label{chapter:fare_system}
この章では,JRの複雑な運賃計算の手順と,本研究が扱う「逆転現象」などの特例について整理する.

\section{片道乗車券}
まずは,本研究で扱う片道乗車券の定義について解説する.
片道乗車券における経路は重複してはならず,環状線を超える経路であってはならない.
ここで,環状線を超える経路とは,発駅から一駅ずつ経路を進めて行った場合に閉路ができ,その先の経路を伸ばしたような経路をいう.
つまり,発着駅が同じとなる閉路の片道乗車券は発売可能である.
発売可能な経路と不可能な経路の形状パターンを図\ref{fig:ticket_shapes}に示す.

\begin{figure}[htbp]
    \centering
    \caption{片道乗車券として発売可能な経路と不可能な経路の形状パターン}
    \label{fig:ticket_shapes}
    \begin{tikzpicture}[
        station/.style={circle, draw=black, thick, minimum size=6mm, inner sep=0pt, fill=white},
        path/.style={->, >={Stealth[length=3mm]}, ultra thick, draw=black!70!black},
        label_text/.style={font=\small\bfseries}
    ]

    % --- 上段タイトル: 発売可能 ---
    % 図の中央付近(x=0.5)の上部(y=3.0)に配置
    \node[font=\large\bfseries] at (-4.0, 3.1) {発売可能な経路};

    % 1. 直線 (Straight)
    \begin{scope}[shift={(-4, 0.3)}]
        \node[station, label=below:発] (A) at (0,0) {A};
        \node[station] (B) at (1.5,0) {B};
        \node[station, label=below:着] (C) at (3,0) {C};
        
        \draw[path] (A) -- (B);
        \draw[path] (B) -- (C);
        
        \node[label_text, above=1.2cm of A] {直線}; 
    \end{scope}

    % 2. 環状線 (Loop) - Oの字
    \begin{scope}[shift={(1.4, 0)}]
        \node[station, label=below:発/着] (A) at (0,-0.5) {A};
        \node[station] (B) at (1.5,-0.5) {B};
        \node[station] (C) at (1.5,1) {C};
        \node[station] (D) at (0,1) {D};
        
        \draw[path] (A) -- (B);
        \draw[path] (B) -- (C);
        \draw[path] (C) -- (D);
        \draw[path] (D) -- (A);

        \node[label_text, above=0.5cm of D] {環状線};
    \end{scope}

    % 3. 6の字 (Lasso) - OK
    \begin{scope}[shift={(5, 0)}]
        \node[station, label=below:発] (A) at (0,-0.5) {A};
        \node[station, label=below:着] (B) at (1.5,-0.5) {B}; 
        \node[station] (C) at (3,-0.5) {C};
        \node[station] (D) at (2.25,1) {D};
        
        \draw[path] (A) -- (B);
        \draw[path] (B) -- (C);
        \draw[path] (C) -- (D);
        \draw[path] (D) -- (B);
        
        \node[label_text, above=2.0cm of A] {6の字}; 
    \end{scope}


    % --- 下段タイトル: 発売不可能 ---
    % 上段と下段の間(y=-3.5)に配置
    \node[font=\large\bfseries] at (-3.6, -2.5) {発売不可能な経路};

    % 4. 折り返し - NG
    \begin{scope}[shift={(-3.5, -5.3)}]
        \node[station, label=below:発/着] (A) at (0,0) {A};
        \node[station] (B) at (2,0) {B};
        
        % B-C間を「行き」と「帰り」の2本のカーブ矢印
        \draw[path] (A) to[bend left=10] (B);
        \draw[path] (B) to[bend left=10] (A);
        
        \node[label_text, above=1.3cm of A] {折り返し};
    \end{scope}

    % 5. T字 (Branch) - NG
    \begin{scope}[shift={(0.3, -6.0)}]
        \node[station, label=below:発] (A) at (0,0) {A};
        \node[station] (B) at (1.5,0) {B};
        \node[station] (C) at (1.5,1.5) {C};
        \node[station, label=below:着] (D) at (3,0) {D};
        
        \draw[path] (A) -- (B);
        
        % B-C間を「行き」と「帰り」の2本のカーブ矢印
        \draw[path] (B) to[bend left=20] (C);
        \draw[path] (C) to[bend left=20] (B);
        
        \draw[path] (B) -- (D);
        
        \node[label_text, above=2.0cm of A] {T字};
    \end{scope}

    % 6. 9の字 (9-shape) - NG
    \begin{scope}[shift={(6.5, -6.0)}]
        \node[station, label=below:発] (A) at (0,0) {A};
        \node[station] (B) at (1.5,0) {B};
        \node[station] (C) at (0.75,1.3) {C};
        \node[station, label=below:着] (D) at (-1.5,0) {D};
        
        \draw[path] (A) -- (B);
        \draw[path] (B) -- (C);
        \draw[path] (C) -- (A);
        \draw[path] (A) -- (D);
        
        \node[label_text, above=2.0cm of D] {9の字};
    \end{scope}

    \end{tikzpicture}
\end{figure}

\section{JRの運賃計算}
6つの旅客鉄道会社から構成される集合をJRと呼ぶこととし,これ以降はそれぞれの会社に対して通称を使う.
表\ref{tab:companies}に各社の通称を示す.

\begin{table}[H]
\centering
\caption{旅客鉄道会社一覧}
\label{tab:companies}
\begin{tabular}{|c|c|} \hline
会社名 & 通称 \\ \hline
北海道旅客鉄道 & JR北海道 \\ \hline
東日本旅客鉄道 & JR東日本 \\ \hline
東海旅客鉄道 & JR東海 \\ \hline
西日本旅客鉄道 & JR西日本 \\ \hline
四国旅客鉄道 & JR四国 \\ \hline
九州旅客鉄道 & JR九州 \\ \hline
\end{tabular}
\end{table}

また,JR東日本・JR東海・JR西日本をまとめて本州3社と呼ぶ.各旅客鉄道会社に旅客営業規則が定められているが,それぞれに内容の差異は無い.
この研究では,JR東日本の旅客営業規則\cite{ryokaku}を参照する.

次に,運賃計算をする手順を以下に示す.
まず,JRの路線は「幹線」と「地方交通線」のいずれかに分類されており,各駅間には0.1km単位で営業キロが設定されている.
運賃は,乗車区間の営業キロの合計に基づき,対応する距離区分(キロ地帯)や賃率を適用して算出される.
本州3社の幹線内相互発着の場合における大人片道普通旅客運賃の計算手順は旅客営業規則 第77条に定められている.
同様にして,本州3社の地方交通線内相互発着の場合における大人片道普通旅客運賃は旅客営業規則 第77条の5に定められている.
よって,幹線は営業キロが240.0kmまで,地方交通線は営業キロが200.0kmまでの運賃表は表\ref{tab:fare_comparison}の通りとなる.

また,幹線と地方交通線を連続して乗車する場合は,地方交通線の乗車区間に対する営業キロにおおよそ1.1倍した値が用いられる.
これを,JR北海道,JR東日本,JR東海,JR西日本では「賃率換算キロ」といい,JR四国,JR九州では「擬制キロ」という.
これらは,地方交通線の各駅間に設定されており,幹線の乗車区間に対する営業キロと地方交通線の乗車区間に対する「賃率換算キロ」及び「擬制キロ」を合算したものを「運賃計算キロ」という.
この運賃計算キロを幹線と営業キロとして,対応する距離区分(キロ地帯)や賃率を適用して算出される.

\begin{table}[htbp]
  \caption{本州3社の幹線と地方交通線における運賃の比較}
  \label{tab:fare_comparison}
  \centering
  \begin{minipage}[t]{0.48\textwidth}
    \centering
    \caption{本州3社の幹線の運賃表}
    \begin{tabular}{cr} \toprule
      営業キロ (km) & 運賃 (円) \\ \midrule
      $\ \ \ \ \ \ \sim \ \ 3.0$   & 150 \\
      $\ \ 3.1 \sim \ \ 6.0$   & 190 \\
      $\ \ 6.1 \sim \ 10.0$  & 200 \\
      $\ 10.1 \sim \ 15.0$ & 240 \\
      $\ 15.1 \sim \ 20.0$ & 330 \\
      $\ 20.1 \sim \ 25.0$ & 420 \\
      $\ 25.1 \sim \ 30.0$ & 510 \\
      $\ 30.1 \sim \ 35.0$ & 590 \\
      $\ 35.1 \sim \ 40.0$ & 680 \\
      $\ 40.1 \sim \ 45.0$ & 770 \\
      $\ 45.1 \sim \ 50.0$ & 860 \\
      $\ 50.1 \sim \ 60.0$ & 990 \\
      $\ 60.1 \sim \ 70.0$ & 1,170 \\
      $\ 70.1 \sim \ 80.0$ & 1,340 \\
      $\ 80.1 \sim \ 90.0$ & 1,520 \\
      $\ 90.1 \sim 100.0$ & 1,690 \\
      $100.1 \sim 120.0$ & 1,980 \\
      $120.1 \sim 140.0$ & 2,310 \\
      $140.1 \sim 160.0$ & 2,640 \\
      $160.1 \sim 180.0$ & 3,080 \\
      $180.1 \sim 200.0$ & 3,410 \\
      $200.1 \sim 220.0$ & 3,740 \\
      $220.1 \sim 240.0$ & 4,070 \\ \bottomrule
    \end{tabular}
  \end{minipage}
  \hfill
  \begin{minipage}[t]{0.48\textwidth}
    \centering
    \caption*{本州3社の地方交通線の運賃表}
    \begin{tabular}{cr} \toprule
      営業キロ (km) & 運賃 (円) \\ \midrule
      $\ \ \ \ \ \ \sim \ \ 3.0$   & 150 \\
      $\ \ 3.1 \sim \ \ 6.0$   & 190 \\
      $\ \ 6.1 \sim \ 10.0$  & 210 \\
      $\ 10.1 \sim \ 15.0$ & 240 \\
      $\ 15.1 \sim \ 20.0$ & 330 \\
      $\ 20.1 \sim \ 23.0$ & 420 \\
      $\ 23.1 \sim \ 28.0$ & 510 \\
      $\ 28.1 \sim \ 32.0$ & 590 \\
      $\ 32.1 \sim \ 37.0$ & 680 \\
      $\ 37.1 \sim \ 41.0$ & 770 \\
      $\ 41.1 \sim \ 46.0$ & 860 \\
      $\ 46.1 \sim \ 55.0$ & 990 \\
      $\ 55.1 \sim \ 64.0$ & 1,170 \\
      $\ 64.1 \sim \ 73.0$ & 1,340 \\
      $\ 73.1 \sim \ 82.0$ & 1,520 \\
      $\ 82.1 \sim \ 91.0$ & 1,690 \\
      $\ 91.1 \sim 100.0$ & 1,880 \\
      $100.1 \sim 110.0$ & 1,980 \\
      $110.1 \sim 128.0$ & 2,310 \\
      $128.1 \sim 146.0$ & 2,640 \\
      $146.1 \sim 164.0$ & 3,080 \\
      $164.1 \sim 182.0$ & 3,410 \\
      $182.1 \sim 200.0$ & 3,740 \\ \bottomrule
    \end{tabular}
  \end{minipage}
\end{table}

これとは別に,私鉄と競合する区間では特定運賃が設けられている.
また,新千歳空港駅,宮崎空港駅,りんくうタウン駅が経路に含まれている経路や児島駅と宇多津駅間の瀬戸大橋を通る経路については通常の運賃計算をした上で加算運賃が発生する.
さらに,旅客営業規則 第140条より,一部区間内を相互発着となるような経路に乗車する場合は,運賃の他に鉄道駅バリアフリー料金が収受される.
今回の研究では,該当の区間相互発着となった場合は鉄道駅バリアフリー料金である10円をあわせた金額を運賃として扱い,実際の切符に印字される金額と出力を一致させる.

\section{対象外とするもの}
今回の研究では,大人の片道乗車券を対象とするが,新幹線を使った経路では乗車券の他に特急券が必要となる上に,複雑な例外処理が必要となるため今回の研究では在来線のみを探索の対象とする.
気仙沼線BRT・大船渡線BRT(バス高速輸送システム)は発売に制限があるため探索の対象外とする.
この研究で扱う片道乗車券の運賃計算経路と運賃は,運賃計算の手順が記載されている旅客営業規則\cite{ryokaku}に則って計算をして,駅の窓口や自動券売機で実際に購入できるものを扱い,ICカード運賃は考慮しない.

\section{分割乗車券}
通常,乗車券は発駅から着駅まで1枚で購入するが,JRの運賃は距離区分によって単価が変動するため,区間を分割して複数枚購入するほうが安価になる場合がある.
これを分割乗車券という.
本研究では,任意の駅間においてこの分割乗車券を自動的に探索し,最安となる組み合わせを提示することを目的とする.

\section{問題定義}
本研究における問題定義は以下の通りである.
発駅と着駅のみを入力として,その間を最も安く移動するような分割乗車券を出力する.
このとき,乗車券の分割数には制限を設けず,最安分割乗車券の候補となる経路は全て探索する.

\section{逆転現象}
運賃は,発駅から着駅までの経路区間に対する運賃計算キロで計算される.
しかし,本研究の主目的である「移動コストの最小化」を達成するためには,単に発着駅間の経路に基づいて運賃を計算するだけでは不十分である.
運賃は,乗車経路の営業キロあるいは運賃計算キロに対して単調非減少であるが,川浦\cite{trrc}が指摘するように,特定の条件下では経路を外方へ延伸することで運賃経路の営業キロあるいは運賃計算キロが短くなり運賃が安くなる「逆転現象」が発生するためである.
この逆転現象が起こった場合,その乗車券は入力された乗車区間のみを使い,乗車しない区間においては権利放棄をして,運賃は乗車経路に対して単調非減少となるようにした.
このとき,券面表示の発駅以外から乗車することを内方乗車,券面表示の着駅以外で下車することを前途放棄と呼ぶ.
本システムでは,以下の事例に代表される逆転現象の特例処理について考慮した実装をし,真の最安運賃を導出できるようにした.
以下で扱う旅客営業規則は付録Aを参照されたい.

\subsection{経路特定区間に関する運賃の逆転現象}
旅客営業規則 第69条で定められている経路特定区間を乗り通す場合は,いずれの経路を利用した場合でも○印の経路で運賃計算が行われる.
このとき,○印のない経路の途中駅からあるいは途中駅までの運賃よりも,経路を外方の分岐駅まで伸ばすことにより,旅客営業規則 第69条を適用させて,運賃経路を○印経由とした方が安い場合がある.
このプログラムでは,経路特定区間○印でない経路上に発着駅があった場合に,その駅を外方の分岐駅まで伸ばした運賃と比較することにより逆転現象が起きないようにした.

図\ref{fig:reverse_route_specific}に,経路特定区間における逆転現象の概要を示す.
通常購入(A→T)では実乗車経路(長い距離)で計算されるが,分岐駅(J)まで購入(A→J)することで経路特定区間が適用され,短い距離で計算されるため,結果として安価になる.

\begin{figure}[htbp]
    \centering
    \label{fig:reverse_route_specific}
    \caption{経路特定区間適用のために延伸する場合の逆転現象}
    \begin{tikzpicture}[
        station/.style={circle, draw=black, thick, minimum size=6mm, inner sep=0pt, fill=white},
        path_line/.style={-, thick, draw=black},
        path_arrow/.style={->, >={Stealth[length=3mm]}, very thick},
        label_text/.style={font=\small\bfseries}
    ]
    
    % 駅の配置
    \node[station, label=below:発駅 A] (A) at (0,0) {};
    \node[station, label=below:目的地 T] (T) at (3,0) {};
    \node[station, label=right:分岐駅 J] (J) at (5,0) {};
    
    % 経路
    \draw[path_line] (A) -- (T);
    \draw[path_line] (T) -- (J);
    
    % 迂回(計算上の最短経路)
    \draw[path_line, dashed] (A) to[bend left=40] node[midway, above] {計算上の最短経路(特定区間)} (J);
    
    % 説明矢印
    \draw[path_arrow, blue] (0,-1) -- node[below] {通常購入:実キロ計算(高)} (3,-1);
    \draw[path_arrow, red] (0,-1.8) -- node[below] {延伸購入:最短キロ計算(安)} (5,-1.8);
    
    \end{tikzpicture}
\end{figure}

\subsection{特定都区市内制度の制度設計に起因する運賃の逆転現象}
旅客営業規則 第86条及び第87条が適用されないような特定都区市内あるいは東京山手線内を発着駅とする経路のうち,旅客営業規則 第86条 及び 第87条 を適用するために,もう一方の駅を外方に伸ばすことで,中心駅からの営業キロが閾値を超えるようにして,運賃を安くすることができる場合がある.
このとき,運賃が安くなった場合は入力された駅ではなく外方の駅までの区間と運賃を出力する.

図\ref{fig:reverse_city_zone}に,特定都区市内制度における逆転現象の例として甲南山手駅から丹治部駅までを最安経路で行く場合を示す.
発着駅が特定都区市内にあり,もう一方の駅までの営業キロが200.0km以下の場合は,発着駅間の営業キロまたは運賃計算キロで計算される.
以下では,姫路駅以遠の姫新線が地方交通線のため運賃計算キロを用いる.
しかし,着駅を外方の岩山駅へ延伸して営業キロ200.0kmを超えさせると,発駅は都区市内中心駅の神戸駅からの計算となる.
このとき,神戸駅が甲南山手駅よりも着駅寄りにあり,計算キロが短縮され運賃が安くなる現象が発生する.
よって,甲南山手駅から丹治部駅までの最安運賃は,神戸市内→岩山の運賃である3,740円であり,丹治部駅から岩山駅までは乗車せず前途放棄する.

\begin{figure}[htbp]
    \centering
    \caption{特定都区市内制度適用のために延伸する場合の逆転現象}
    \label{fig:reverse_city_zone}
    \begin{tikzpicture}[
        station/.style={circle, draw=black, thick, minimum size=6mm, inner sep=0pt, fill=white},
        zone/.style={draw=gray, dashed, rounded corners, fill=gray!10},
        path_line/.style={-, thick, draw=black},
        path_arrow/.style={->, >={Stealth[length=3mm]}, very thick},
        label_text/.style={font=\small\bfseries}
    ]
    
    % 特定都区市内ゾーン
    \filldraw[zone] (-1,-1) rectangle (3,2.5);
    \node at (1, 2) {\bf 神戸市内};
    
    % 駅の配置
    \node[station, label=below:甲南山手] (A) at (0,0) {};
    \node[station, label=below:神戸, fill=black!20] (C) at (2,0) {};
    \node[station, label=below:姫路] (H) at (6,0) {};
    \node[station, label=below:丹治部] (T) at (10,0) {};
    \node[station, label=below:岩山] (E) at (12,0) {};
    
    \node at (4, 0.3) {\small 山陽本線};
    \node at (9, 0.3) {\small 姫新線};
    % 経路
    \draw[path_line] (A) -- (C) -- (H) -- (T) -- (E);
    
    % 距離の比較
    \draw[path_arrow] (0, 1) -- node[above] {  運賃計算キロ:226.8km} (10, 1);
    \draw[path_arrow] (2, -1.2) -- node[below] {運賃計算キロ:219.6km  } (12, -1.2);
    
    % 閾値ライン
    \draw[dashed] (11, -0.4) -- (11, 1.7) node[above] {神戸駅からの営業キロ200.0km境界};
    
    \end{tikzpicture}
\end{figure}

%%%%%%%%%%%%%%%%%%%%%
% 3章
%%%%%%%%%%%%%%%%%%%%%
\chapter{最安運賃計算プログラム} \label{chapter:program}

後述する分割乗車券プログラムから呼び出されるプログラムであり,入力された経路を乗車することができる片道乗車券の中で最安となりうる経路全ての運賃計算を行い,最安だった経路と運賃を出力する.

運賃計算アルゴリズムは以下の通りである.
\begin{itemize}
    \item 並行する他社路線に対抗するため,通常の距離制運賃よりも安く設定された区間である特定運賃区間と経路が一致する場合はその特定運賃を返す.
    \item 入力された経路が,山手線内の駅相互発着の場合,東京附近における電車特定区間内相互発着の場合,大阪附近における電車特定区間内相互発着の場合はそれぞれに対応した賃率により運賃を計算する.
    \item 全ての駅間を旅客会社ごとに分けて,まずはすべての区間を本州3社の運賃体系で計算をする.
    JR北海道,JR四国,JR九州のそれぞれの区間で,会社ごとの運賃から本州3社の運賃の差を加算額として加える.
    また,加算運賃区間にまたがる場合は加算運賃を加えた額を返す.
\end{itemize}

運賃計算をした上で,鉄道バリアフリー料金該当区間相互発着の場合は鉄道バリアフリー料金を加えた額を運賃とする.\\

この運賃計算プログラムでは,旅客営業規則の通りに,入力された経路が計算される.
しかし,その経路の始発駅や終着駅を外方の駅として運賃経路を伸ばしたときに運賃が安くなる場合は,伸ばした駅を乗車券の始発駅や終着駅として安い運賃を採用する.
ここでは,分割することは考えずに入力された経路を内包し,最も安く乗車することができる1枚の乗車券が出力される.

この運賃計算方法により,1枚の乗車券と分割乗車券のいずれを考えたときでも,経路を外方へ伸ばしたときに運賃が安くなることは無い.
よって,運賃は運賃計算キロに対して単調非減少であることが示せる.

%%%%%%%%%%%%%%%%%%%%%
% 4章
%%%%%%%%%%%%%%%%%%%%%
\chapter{分割乗車券探索アルゴリズム} \label{chapter:algorithm}

本研究では,運賃が運賃計算キロに対して単調非減少であるという前提に基づき,探索対象を単純パスに限定する.
経路に閉路が含まれる場合,その閉路を取り除いた経路の総距離は必ず元の経路よりも短くなるか等しくなる.
運賃の単調非減少性より,閉路を含む経路が閉路を含まない経路よりも安くなることはない.
同様に,経路の一部を折り返すような経路については,折り返し地点で乗車券を分割する必要があるが,重複区間を取り除いた単純な経路と比較して運賃が安くなることはない.
したがって,最安解となる経路の候補は単純パスの中に必ず存在する.

また,探索アルゴリズムの選定において,A*アルゴリズムなどのヒューリスティック探索を用いることで計算量を削減する手法が考えられる.
しかし,JRの運賃制度には営業キロが物理的な距離よりも短く設定されている区間が存在する.
このため,物理的な距離等をヒューリスティック関数 $h(n)$ として用いた場合,推定値が実際のコストを上回り過大評価するケースが発生し,A*アルゴリズムの許容条件 $h(n) \leqq h^*(n)$ を満たさなくなる.
これにより,最安解の見逃しが発生するリスクがあるため,本研究ではヒューリスティックを用いない Dijkstra's Algorithm および Yen's Algorithm を採用し,解の正当性を保証する.

提案手法の手順は以下の通りである.

\begin{enumerate}
    \item \textbf{基準経路の探索:}\\
    Dijkstra's algorithmを用い,発駅から着駅までの運賃計算キロが最短となる経路を探索する.
    これを探索の基準経路とし,その正確な運賃計算キロ(0.1km単位)を $d_{min}$,その運賃を $C_{base}$ とする.

    \item \textbf{候補経路の列挙:}\\
    Yen's Algorithmを用い,第$K$最短経路を順次探索して候補経路リストに追加する.
    ここで,探索の打ち切り条件(距離の閾値 $d_{limit}$)を次のように定義する.
    本研究では,基準経路の距離 $d_{min}$ 以下の範囲で成立しうる最安キロ単価を用いる.
    
    距離 $x$ における運賃を $F(x)$ としたとき,距離 $d$ 以下の範囲における最安キロ単価関数 $u_{best}(d)$ を以下のように定義する.
    \[u_{best}(d) = \min_{x \in S, x \leqq d} \left( \frac{F(x)}{x} \right)\]
    ここで,$S$ は運賃表において運賃が変化する距離の境界値(3.0, 6.0, 10.0, $\dots$)および $d$ 自身を含む集合である.また,$F(x)$ は消費税および端数処理を含んだ実際の運賃表の値とする.
    
    この $u_{best}(d_{min})$ を用いて,探索打ち切り距離 $d_{limit}$ を以下の式で定める.
    \[d_{limit} = \frac{C_{base}}{u_{best}(d_{min})} + \alpha\]
    
    ここで,$\alpha$ は特定都区市内制度による距離の乖離を吸収するための補正項であり,以下の式で算出される.
    \[\alpha = \delta_{start} + \delta_{end}\]
    ただし,$\delta_{start}$ および $\delta_{end}$ は,それぞれ発駅および着駅が特定都区市内に属する場合の「当該駅からその特定都区市内の中心駅までの運賃計算キロ」とし,属さない場合は 0 とする.
    探索経路の運賃計算キロが $d_{limit}$ を超えた時点で,それ以上の探索を打ち切る.

    \item \textbf{解の出力:}\\
    列挙された各候補経路について,全てパターンの分割運賃計算を行い最安分割運賃と経路を出力する.
\end{enumerate}

%%%%%%%%%%%%%%%%%%%%%
% 5章
%%%%%%%%%%%%%%%%%%%%%
\chapter{探索システムの設計と実装} \label{chapter:implementation}

\section{開発環境と使用技術}
\begin{itemize}
    \item TypeScript(Node.js)
\end{itemize}

\section{運賃・経路データの構築}
以下のデータをそれぞれプログラムへ読み込む.
なお,営業キロと賃率換算キロ及び擬制キロはJR時刻表\cite{jikoku}のデータを利用している.

\begin{itemize}
    \item 入力用データリスト
    \begin{itemize}
        \item 駅名かなデータ,路線駅データ,乗換データ
    \end{itemize}
    \item 特定市内データ
    \begin{itemize}
        \item 市名とそれに属する駅の集合
    \end{itemize}
    \item 経由印字データリスト
    \begin{itemize}
        \item 路線名に紐づく経由印字
    \end{itemize}
    \item 各駅間におけるセグメントデータリスト
    \begin{itemize}
        \item 営業キロ,換算キロ,旅客会社
    \end{itemize}
    \item 私鉄との競合区間における特定運賃リスト
    \item 旅客営業規則 第69条による経路の置き換えリスト
    \item 山手線内の駅リスト
\end{itemize}

\subsection{主要モジュールの実装}

\subsubsection{運賃計算プログラム}
分割乗車券プログラムから呼び出されるプログラムであり,発駅と着駅の他に経路も指定する必要がある.
入力として受け取った発駅から着駅までの経路を含む乗車券のうち最安のものを返す.
ここでは,分割することは考えずに1枚の乗車券として出力する.
ただし,このときに運賃計算経路を外方に伸ばした時に運賃が安くなる場合は乗車券の区間と運賃を伸ばした経路に置き換え,増運賃区間にまたがる場合や鉄道バリアフリー料金の10円も加えた運賃を返す.

\subsubsection{分割乗車券プログラム}
本節では,第3章で定義した探索アルゴリズムを,TypeScriptを用いて具体的なソフトウェアとして実装する際の詳細について述べる.
実装においては,計算速度とメモリ効率を重視し,以下のデータ構造および処理方式を採用した.

\begin{enumerate}
    \item \textbf{Priority Queueの実装:}\\
    基準経路探索およびYen's Algorithmの内部処理で用いるDijkstra法の実装においては,探索ノードの取り出しを高速化するためにBinary Heapを用いたPriority Queueを採用した.
    これにより,ノード数が膨大になる長距離区間の探索においても,計算量の増大を抑えている.

    \item \textbf{Yen's Algorithmにおける経路管理:}\\
    第$K$最短経路の探索において,候補となる経路群はメモリ上でリスト構造として管理される.
    第3章で定義した打ち切り閾値 $d_{limit}$ の判定は,経路が1本見つかるごとに動的に行われ,条件を満たさなくなった時点で直ちにループを脱出する設計とした.
    これにより,無駄な探索処理が走ることを防いでいる.
\end{enumerate}

%%%%%%%%%%%%%%%%%%%%%
% 6章
%%%%%%%%%%%%%%%%%%%%%
\chapter{評価実験と考察} \label{chapter:evaluation}

\section{実験概要}
本システムの有用性を検証するため,以下の3つの観点から評価実験を行う.
\begin{enumerate}
    \item \textbf{正当性:} 既存の分割計算サービスや手計算による正解データと,本システムの出力が一致するか.特に第2章で述べた特例区間において正しい挙動を示すか.
    \item \textbf{有用性:} 分割乗車券を利用しない場合と比較して,どの程度の運賃削減効果が得られるか.
    \item \textbf{性能:} 実用的なWebサービスとして運用可能な応答時間で解を導出できるか.
\end{enumerate}

評価に用いた計算機のスペック:MacBook Pro, Apple M4, 16GB

\section{計算結果の正当性評価}
無作為に選んだ2つの駅間の他に,特例が適用される経路を,以下のようにして正当性を確認した.
\begin{enumerate}
    \item \textbf{分割をしない通常の運賃}\\
        駅の自動券売機や運賃計算のWebサービスで経路を入力して表示されるきっぷの運賃と,私の作成した運賃計算プログラムの運賃が等しくなることを確認した.
        各特例が適用されていることに加えて,特定の経路を入力した場合,特例が適用するために外方へ伸ばされ安くなることも確認できた.
        また,インターネット上の個人ブログ等で紹介されている特殊な経路の運賃も,私の作成した運賃計算プログラムと同額となることを確認した.
    
    \item \textbf{分割を考慮した最安運賃}\\
        私の作った任意の発駅と着駅のみの入力で最安分割運賃を計算するプログラムは,現在私の作ったプログラムの他に存在しない.
        そのため,任意の駅間での最短運賃計算キロの経路を利用して,oba氏の乗車券分割プログラムと比較することにした.
        例として,東京駅と仙台駅間における最短運賃計算キロの経路は東北本線であり,oba氏の乗車券分割プログラムにおいてその経路では5,830円が出力される.
        しかし,私の分割乗車券プログラムでは,東北本線中の赤羽駅と大宮駅間において埼京線を使って迂回することにより,5,800円を出力する.
        この迂回経路をoba氏の乗車券分割プログラムに入力すると,5,800円となり最安運賃が一致することを確認した.
    
\end{enumerate}

\section{性能評価}
←いくつかのテストケースをグラフにしてここに貼ります.→

\section{考察}
←計算時間がかかる原因や対策を考えます.→

%%%%%%%%%%%%%%%%%%%%%
% 7章
%%%%%%%%%%%%%%%%%%%%%
\chapter{結論} \label{chapter:conclusion}
\section{本研究の総括}
本研究では,JR分割乗車券の最適解を探索するプログラムをDijkstra's algorithmを用いて開発した.

\section{将来の展望と今後の課題}
今後の展望としては,新幹線や特急料金を含めた最適化や,Webサービスとしての一般公開,モバイルアプリ化などが挙げられる.
一方で課題としては,運賃改定時のシステム再構築の容易化や,さらなる計算速度の向上,各種割引制度への対応が残されている.

%%%%%%%%%%%%%%%%%%%%%
% 謝辞
%%%%%%%%%%%%%%%%%%%%%
\syaji
\par
本論文について貴重な御助言を頂いた指導教員や研究室のメンバー方には心より感謝致します.

%%%%%%%%%%%%%%%%%%%%%
% 参考文献
%%%%%%%%%%%%%%%%%%%%%
\begin{thebibliography}{99}
\addcontentsline{toc}{chapter}{参考文献}

\bibitem{dijkstra}
E.W. Dijkstra, A note on two problems in connexion with graphs, Numerische Mathematik, vol.~1, pp.~269--271, 1959.

\bibitem{yen}
Jin Y. Yen, Finding the K Shortest Loopless Paths in a Network, Management Science, vol.~17, no.~11, pp.~712--716, 1971.

\bibitem{oba}
oba,乗車券分割プログラム,入手先〈\url{https://bunkatsu.jp/}〉(参照2025-12-25).

\bibitem{hirata}
平田 直也, 中桐 斉之, スマートフォンに特化した乗車券分割アプリの開発, 第78回全国大会講演論文集, vol.~2016, no.~1, pp.~387--388, 2016.

\bibitem{ryokaku}
JR東日本,旅客営業規則,入手先〈\url{https://www.jreast.co.jp/ryokaku/}〉(参照2025-12-25).

\bibitem{jikoku}
時刻表編集部,JR時刻表 2025年10月号,交通新聞社,2025.

\bibitem{trrc}
川浦 龍一,遠い方が安くなる?! なぜなにJR旅客営業制度 --きっぷのふしぎ--,交通法規研究会,東京,2024.

\end{thebibliography}

%%%%%%%%%%%%%%%%%%%%%
% 付録
%%%%%%%%%%%%%%%%%%%%%
\appendix
\def\labelenumi{(\theenumi)}
\chapter{運賃計算における経路補正}

片道乗車券の運賃計算する際の経路補正に関する規則のみを抜粋して掲載する.規則の全文については参考文献\cite{ryokaku}を参照されたい.
\\\\
\noindent\textbf{(旅客運賃・料金計算上の経路等)}\\
\noindent\textbf{第67条}\\
旅客運賃・料金は、旅客の実際乗車する経路及び発着の順序によって計算する。\\

\noindent\textbf{(特定区間における旅客運賃・料金計算の営業キロ又は運賃計算キロ)}\\
\noindent\textbf{第69条}\\
第67条の規定にかかわらず、次の各号に掲げる区間の普通旅客運賃・料金は、その旅客運賃・料金計算経路が当該各号末尾のかっこ内の両線路にまたがる場合を除いて、○印の経路の営業キロ(第9号については運賃計算キロ。ただし、岩国・櫛ヶ浜間相互発着の場合にあっては営業キロ)によって計算する。この場合、各号の区間内については、経路の指定を行わない。

\begin{enumerate}
    \item 大沼以遠(仁山方面)の各駅と、森以遠(石倉方面)の各駅との相互間\\
        \qquad    
        $
            \Bigg(
                \begin{tabular}{l}
                     東森駅経由函館本線\\
                    ○大沼公園駅経由函館本線
                \end{tabular}
            \Bigg)
        $
    \\
    \item 日暮里以遠(鶯谷又は三河島方面)の各駅と、赤羽以遠(川口、北赤羽又は十条方面)の各駅との相互間\\
        \qquad    
        $
            \Bigg(
                \begin{tabular}{l}
                     尾久経由東北本線\\
                    ○王子経由東北本線
                \end{tabular}
            \Bigg)
        $
    \\
    \item 赤羽以遠(尾久、東十条又は十条方面)の各駅と、大宮以遠(土呂、宮原又は日進方面)の各駅との相互間\\
        \qquad    
        $
            \Bigg(
                \begin{tabular}{l}
                     戸田公園・与野本町経由東北本線\\
                    ○川口・浦和経由東北本線
                \end{tabular}
            \Bigg)
        $
    \\
    \item 品川以遠(高輪ゲートウェイ又は大崎方面)の各駅と、鶴見以遠(新子安、国道又は羽沢横浜国大方面)の各駅との相互間\\
        \qquad    
        $
            \Bigg(
                \begin{tabular}{l}
                     西大井経由東海道本線\\
                    ○大井町経由東海道本線
                \end{tabular}
            \Bigg)
        $
    \\
    \item 東京以遠(有楽町又は神田方面)の各駅と、蘇我以遠(鎌取又は浜野方面)の各駅との相互間\\
        \qquad    
        $
            \Bigg(
                \begin{tabular}{l}
                     京葉線\\
                    ○総武本線・外房線
                \end{tabular}
            \Bigg)
        $
    \\
    \item 山科以遠(京都方面)の各駅と、近江塩津以遠(新疋田方面)の各駅との相互間\\
        \qquad    
        $
            \Bigg(
                \begin{tabular}{l}
                     東海道本線・北陸本線\\
                    ○湖西線
                \end{tabular}
            \Bigg)
        $
    \\
    \item 大阪以遠(塚本又は新大阪方面)の各駅と、天王寺以遠(東部市場前又は美章園方面)の各駅との相互間\\
        \qquad    
        $
            \Bigg(
                \begin{tabular}{l}
                     福島経由大阪環状線\\
                    ○天満経由大阪環状線
                \end{tabular}
            \Bigg)
        $
    \\
    \item 三原以遠(糸崎方面)の各駅と、海田市以遠(向洋方面)の各駅との相互間\\
        \qquad    
        $
            \Bigg(
                \begin{tabular}{l}
                     呉線\\
                    ○山陽本線
                \end{tabular}
            \Bigg)
        $
    \\
    \item 岩国以遠(和木方面)の各駅と、櫛ヶ浜以遠(徳山方面)の各駅との相互間\\
        \qquad    
        $
            \Bigg(
                \begin{tabular}{l}
                     山陽本線\\
                    ○岩徳線
                \end{tabular}
            \Bigg)
        $

\end{enumerate}

\noindent\textbf{第70条}\\
第67条の規定にかかわらず、旅客が次に掲げる図の太線区間を通過する場合の普通旅客運賃・料金は太線区間内の最も短い営業キロによって計算する。この場合、太線内は、経路の指定を行わない。\\
\newline
\includegraphics[scale=0.9]{70.png}
\newline

\noindent\textbf{2}\\
蘇我以遠(鎌取又は浜野方面)の各駅と前条第1項第5号に掲げるいずれかの経路を経由して前項に掲げる図の太線区間を大久保以遠(東中野方面)、三河島以遠(南千住方面)、川口以遠(西川口方面)又は北赤羽以遠(浮間舟渡方面)へ通過する場合の普通旅客運賃・料金は、第67条及び前条第1項第5号の規定にかかわらず、外房線蘇我・千葉間、総武本線千葉・錦糸町間及び前項に掲げる図の太線区間内の最も短い経路の営業キロによって計算する。\\

\noindent\textbf{(特定都区市内にある駅に関連する片道普通旅客運賃の計算方)}\\
\noindent\textbf{第86条}\\
次の各号の図に掲げる東京都区内、横浜市内(川崎駅、尻手駅、八丁畷駅、川崎新町駅及び小田栄駅並びに鶴見線各駅を含む。)、名古屋市内、京都市内、大阪市内(南吹田駅、高井田中央駅、JR河内永和駅、JR俊徳道駅、JR長瀬駅及び衣摺加美北駅を含む。)、神戸市内(道場駅を除く。)、広島市内(海田市駅及び向洋駅を含む。)、北九州市内、福岡市内(姪浜駅、下山門駅、今宿駅、九大学研都市駅及び周船寺駅を除く。)、仙台市内又は札幌市内(以下これらを「特定都区市内」という。)にある駅と、当該各号に掲げる当該特定都区市内の◎印の駅(以下「中心駅」という。)から片道の営業キロが200キロメートルを超える区間内にある駅との相互間の片道普通旅客運賃は、当該中心駅を起点又は終点とした営業キロ又は運賃計算キロによって計算する。\\
ただし、特定都区市内にある駅を発駅とする場合で、普通旅客運賃の計算経路が、その特定都区市内の外を経て、再び同じ特定都区市内を通過するとき、又は特定都区市内にある駅を着駅とする場合で、発駅からの普通旅客運賃の計算経路が、その特定都区市内を通過して、その特定都区市内の外を経るときを除く。\\[5pt]
\quad(1)〜(10) 略\\

\noindent\textbf{(東京山手線内にある駅に関連する片道普通旅客運賃の計算方)}\\
\noindent\textbf{第87条}\\
東京山手線内にある駅と、中心駅から片道の営業キロが100キロメートルを超え200キロメートル以下の区間内にある駅との相互間の片道普通旅客運賃は、当該中心駅を起点又は終点とした営業キロ又は運賃計算キロによって計算する。\\
ただし、東京山手線内にある駅を発駅とする場合で、普通旅客運賃の計算経路が、東京山手線内の外を経て、再び東京山手線内を通過するとき、又は東京山手線内にある駅を着駅とする場合で、発駅からの普通旅客運賃の計算経路が、東京山手線内を通過して、東京山手線内の外を経るときを除く。\\

\noindent\textbf{(新大阪駅又は大阪駅発又は着となる片道普通旅客運賃の計算方)}\\
\noindent\textbf{第88条}\\
新大阪駅又は大阪駅と姫路駅以遠(英賀保、京口又は播磨高岡方面)の各駅との相互間の片道普通旅客運賃は、姫路駅を経由する場合に限り、大阪駅を起点又は終点とした営業キロ又は運賃計算キロによって計算する。\\

\noindent\textbf{(北新地駅発又は着となる片道普通旅客運賃の計算方)}\\
\noindent\textbf{第89条}\\
北新地駅と尼崎以遠(立花又は塚口方面)の各駅との相互間の片道普通旅客運賃は、加島駅を経由する場合に限り、大阪駅を起点又は終点とした営業キロ又は運賃計算キロ(いずれも塚本駅を経由するものとする。)によって計算する。ただし、第86条の規定により片道普通旅客運賃を計算する場合を除く。\\

\noindent\textbf{(鉄道駅バリアフリー料金)}\\
\noindent\textbf{第140条}\\
次の各号に掲げる区間内相互発着となる区間に乗車する場合は、鉄道駅バリアフリー料金を収受する。
\begin{enumerate}
    \item 第78条第2項第1号に定める東京附近における電車特定区間及び第80条の規定を適用する区間(同条第1項第1号から第4号の区間にかかるものに限る。)
    \item 第78条第2項第2号に定める大阪附近における電車特定区間及び第80条の規定を適用する区間(同条第1項第5号から第14号及び同条第2項の区間にかかるものに限る。)
    \item 東海道本線(新幹線)中豊橋・岐阜羽島間、東海道本線中豊橋・大垣間、武豊線、中央本線中多治見・名古屋間、関西本線中名古屋・四日市間(ただし、対象区間のみを経由して乗車する場合に限る。)
\end{enumerate}

\noindent\textbf{2}\\
前項の規定により収受する鉄道駅バリアフリー料金は、次の各号に定めるとおりとする。
\begin{enumerate}
    \item 前項第1号に掲げる区間内相互発着となる区間に乗車する場合\\
    イ 大人片道普通旅客運賃とあわせ収受する額\\
      片道乗車あたり10円
    \item 前項第2号及び第3号に掲げる区間内相互発着となる区間に乗車する場合\\
    イ 大人片道普通旅客運賃とあわせ収受する額\\
      片道乗車あたり10円
\end{enumerate}

\end{document}