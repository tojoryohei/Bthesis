\documentclass[a4j,12pt]{jreport}
%\documentclass{jreport}
\usepackage[dvipdfmx]{graphicx}
\usepackage{amsmath,amssymb}
\usepackage{here}
\usepackage{algorithm}
\usepackage{algpseudocode}
\usepackage{hhline}
\usepackage[hang,small,bf]{caption}
\usepackage[subrefformat=parens]{subcaption}
\usepackage{url}
\captionsetup{compatibility=false}

\def\syaji{ \chapter*{謝辞} \addcontentsline{toc}{chapter}{謝辞}}
\renewcommand{\bibname}{参考文献}
\setlength{\textheight}{\paperheight}
\setlength{\topmargin}{4.6mm}
\addtolength{\topmargin}{-\headheight}
\addtolength{\topmargin}{-\headsep}
\addtolength{\topmargin}{-\headheight}
\addtolength{\textheight}{-60mm}

\setlength{\textwidth}{\paperwidth}
\setlength{\oddsidemargin}{-0.4mm}
\setlength{\evensidemargin}{-0.4mm}
\addtolength{\textwidth}{-50mm}

\begin{document}

%%%%%%%%%%%%%%%%%%%%%
% 表紙
%%%%%%%%%%%%%%%%%%%%%
\thispagestyle{empty}
\begin{center}
  \begin{Large}
    \vspace*{0.7cm}
    {\large 卒業研究論文}\\
    \vspace*{2.5cm}
    {\LARGE\bf JR最安分割乗車券探索システムの開発}\\
    \vspace*{7.5cm}
    東條 涼平\\
    学籍番号\hspace*{1zw}22D8102015D\\
    \vspace*{2.5cm}
    中央大学理工学部情報工学科\hspace*{1zw} アルゴリズム理論基礎研究室\\
    \vspace*{3.0cm}
    2026年3月\\
  \end{Large}
\end{center}


%%%%%%%%%%%%%%%%%%%%%
% 概要
%%%%%%%%%%%%%%%%%%%%%
\newpage
\renewcommand{\baselinestretch}{1.25} \selectfont
\pagenumbering{roman}


\begin{center} {\large \bf{概 要}} \end{center}

JR線を利用する際,任意の駅間において,1枚で購入するきっぷの運賃よりも乗車区間を区切って購入するきっぷのそれぞれの運賃の合計のほうが安い場合がある.この区切った乗車券を分割乗車券という.この最安解を出すためには,従来では指定した経路ですべての分割パターンで運賃を比較する必要があるが,このとき分割乗車券の経路が最短経路でない場合もあるため,複数の経路を試す必要があった.そこで,原則として遠くへ行ったときに運賃が上がらないことを利用して,ダイクストラ法(および,A*アルゴリズム)を使って,経路を指定せずに発駅と着駅の入力のみで最適解を出すプログラムを作成した.

\vspace{1zw} \noindent
{\bf キーワード: }分割乗車券,経路探索,A*アルゴリズム,JR運賃,グラフ理論

%%%%%%%%%%%%%%%%%%%%%
% 目次
%%%%%%%%%%%%%%%%%%%%%
\tableofcontents


\newpage
\pagenumbering{arabic}

%%%%%%%%%%%%%%%%%%%%%
% 1章
%%%%%%%%%%%%%%%%%%%%%
\chapter{序論} \label{chapter:intro}

\section{研究の背景}
JRでは営業キロを一定の幅に区切って運賃を設定しているため,分割して購入したほうが安くなる時がある.

出発駅と到着駅のほかに経路を指定した場合にその経路内に分割駅分割乗車券プログラムは存在するが,旅客が最適解の経路を特定することは困難である.これは,都心のように路線が狭い範囲で密集している場合など,最短経路上に分割駅が無い場合もあるからである.

\section{研究の対象}
大人(および小人)の片道乗車券(および通期定期券)を計算の対象とする.

運賃計算には多くの例外があり複雑である.また,日本語の解釈によって運賃が異なることもある.この研究で扱うきっぷの運賃と経路は,運賃計算の手順が記載されている旅客営業規則に則り計算をして,駅の窓口や自動券売機で実際に購入できるものを扱うこととする.

\section{関連研究・従来の手法}
分割乗車券プログラム(oba氏による)などが存在する.

\section{研究の目的と意義}
旅客らに実用的な節約手段を提供でき,複雑な制約を持つ大規模グラフ問題に対する効率的な解法を高速な計算時間で実装する.

%%%%%%%%%%%%%%%%%%%%%
% 2章
%%%%%%%%%%%%%%%%%%%%%
\chapter{JR運賃制度と問題の定義} \label{chapter:fare_system}

\section{JR運賃計算の基礎}
JRとは6つの旅客鉄道会社と1つの貨物鉄道会社から構成される.この研究では旅客鉄道会社のみを扱い,これ以降は通称を使う.表\ref{tab:jr_companies}に各社の通称を示す.

\begin{table}[H]
\centering
\caption{旅客鉄道会社一覧}
\label{tab:jr_companies}
\begin{tabular}{|c|c|} \hline
会社名 & 通称 \\ \hline
北海道旅客鉄道 & JR北海道 \\ \hline
東日本旅客鉄道 & JR東日本 \\ \hline
東海旅客鉄道 & JR東海 \\ \hline
西日本旅客鉄道 & JR西日本 \\ \hline
四国旅客鉄道 & JR四国 \\ \hline
九州旅客鉄道 & JR九州 \\ \hline
\end{tabular}
\end{table}

旅客営業規則では\textbf{乗車券}は以下のとおりに定められている.
\begin{itemize}
    \item 普通乗車券
    \begin{itemize}
        \item 片道乗車券
        \item 往復乗車券
        \item 連続乗車券
    \end{itemize}
    \item 定期乗車券
    \begin{itemize}
        \item 通勤定期乗車券
        \item 通学定期乗車券
        \item 特殊定期乗車券
        \begin{itemize}
            \item 特別車両定期乗車券
            \item 特殊均一定期乗車券
        \end{itemize}
    \end{itemize}
    \item 普通回数乗車券
    \item 団体乗車券
    \item 貸切乗車券
\end{itemize}

今回の研究では実用的な利用を想定しているため,普通乗車券(と通勤定期乗車券)のみを扱う.(学割や株主優待などの割引を適用するケースも用意したい.)

JRの運賃計算で使われる駅間はそれぞれの駅間は以下の情報を持つ.
\begin{itemize}
    \item 「幹線」か「地方交通線」のどちらか
    \item 営業キロ
    \item 換算キロ(地方交通線のみ)
    \item 所属する旅客会社
\end{itemize}
これらの情報をもとに運賃計算がなされる.

一部区間を相互発着となる区間に乗車する場合は,運賃の他に\textbf{鉄道駅バリアフリー料金}が収受される.今回の研究では,運賃に鉄道駅バリアフリー料金をあわせた金額を運賃とする.

\section{問題の定式化}
分割乗車券プログラムを開発するにあたって,運賃計算プログラムを作る必要がある.ICカードの運賃や途中下車をすることは考慮しない.この運賃計算プログラムでは,与えられた経路の運賃を出力するだけでなく,入力された経路の始発駅や終着駅を外方の駅としたときに運賃が安くなる場合,その駅をきっぷの始発駅や終着駅とする.その区間の一部を使いそのきっぷの運賃を入力区間の運賃とする.これを内方乗車と呼ぶ.(このとき,特例による経路の置き換えは行わない.)

%%%%%%%%%%%%%%%%%%%%%
% 3章
%%%%%%%%%%%%%%%%%%%%%
\chapter{探索手法} \label{chapter:algorithm}

\section{探索アルゴリズムの選定}
ダイクストラ法とA*アルゴリズムを実装して比較をする.
ノード数が膨大なため,ヒューリスティックを用いないダイクストラ法では探索空間が広すぎると考えている.

\section{A*アルゴリズムによる最適解探索}
評価関数 $f(n) = g(n) + h(n)$ の各項が本問題においては以下のようになる.
\begin{itemize}
    \item $g(n)$: 出発駅からノード$n$までの最安分割運賃
    \item $h(n)$: ノード$n$からゴールまでの推定最安運賃
\end{itemize}

\section{ヒューリスティック関数の設計}
緯度と経路から導き出された直線距離と最安キロ単価から求められた推定最安運賃を用いる.

%%%%%%%%%%%%%%%%%%%%%
% 4章
%%%%%%%%%%%%%%%%%%%%%
\chapter{システムの実装} \label{chapter:implementation}

\section{開発環境と使用技術}
\begin{itemize}
    \item TypeScript(Node.js)
\end{itemize}

\section{運賃・経路データの構築}
以下のデータをそれぞれJSON形式からプログラムにシングルインスタンスとして読み込む.

\begin{itemize}
    \item 入力用データリスト
    \begin{itemize}
        \item 駅名カナデータ,路線駅データ,乗換データ
    \end{itemize}
    \item 特定市内データ
    \begin{itemize}
        \item 名前(例:札幌市内),駅の集合
    \end{itemize}
    \item 経由印字データリスト
    \begin{itemize}
        \item 路線名と経由印字
    \end{itemize}
    \item セグメントデータリスト
    \begin{itemize}
        \item 営業キロ,換算キロ,旅客会社
    \end{itemize}
    \item 特定運賃リスト
    \item 経路の置き換えリスト
    \item 山手線内データリスト
\end{itemize}

\section{主要モジュールの実装}

\subsection{分割乗車券プログラム}
\begin{enumerate}
    \item 出発駅から隣駅までの経路と運賃をキューに入れる.
    \item キューの中で最安のものを1つ選ぶ.
    \item 2で選んだ駅から隣駅までの(通しの)運賃,及び,その隣駅から出発駅までの各駅における始発駅からの運賃とその隣駅までの運賃の和の中で最安の経路と運賃を計算する.このときに,片道乗車券における以下の2つのルールを守らなくてはならない.
    \begin{itemize}
        \item 環状線を超える経路であってはならない.
        \item 経路が重複してはならない.
    \end{itemize}
    ただ,環状線を超える経路,つまり6の字になるような経路だけが最安の経路になることはないため,単純に同じ駅が2回出てこなければ良い.(遠いほうが安いことはないので6の字の経路の運賃 $\ge$ 環状部分を取り除いた経路が成り立つ)
    \item 2で選んだ駅が到着駅であったら計算結果を返して終了する.そうでなければ計算結果をキューに入れて2へ戻る.
\end{enumerate}

\subsection{運賃計算プログラム}
分割乗車券プログラムから呼び出される.
\begin{itemize}
    \item 入力された経路を一駅ごとに変換する.
\end{itemize}

\begin{quote}
【例】\\
池袋 (山手線) 新宿 (中央線) 中野\\
$\downarrow$(変換)\\
池袋 (山手線) 目白 (山手線) 高田馬場 (山手線) 新大久保 (山手線) 新宿 (中央線) (中)大久保 (中央線) 東中野 (中央線) 中野
\end{quote}

\begin{itemize}
    \item 特定運賃区間と一致する場合はその運賃を返す.
    \item セグメントデータを旅客会社ごとに分けて,まずはすべての区間をJR東日本・JR東海・JR西日本(以降これらを本州三社と呼ぶ)の運賃体系で計算をする.北海道・四国・九州ごとの区間でその会社ごとの運賃から本州三社の運賃の差を加算運賃として加える.また,増運賃区間にまたがる場合や鉄道バリアフリー料金も加えた運賃を返す.ただし,このときに運賃計算経路を外方に伸ばした時に運賃が安くなる場合は乗車券の区間と運賃をそれに置き換える.
\end{itemize}

%%%%%%%%%%%%%%%%%%%%%
% 5章
%%%%%%%%%%%%%%%%%%%%%
\chapter{評価実験と考察} \label{chapter:evaluation}

本章では、実装したプログラムが目的を達成できたか(正しく、速く計算できるか)を評価する。

\section{実験概要}
評価に用いた計算機のスペック、テストケース(例:短距離、長距離、特定区間を含むなど)の選定基準について述べる。

\section{計算結果の正当性評価}
プログラムが算出した解が、本当に「最適(最安)」であるかを検証する。
(例:既知の分割パターンや、他の信頼できるサービスとの結果比較。)

\section{性能評価(計算速度)}
様々なテストケースにおいて、最適解を算出するまでに要した時間を計測し、結果を表やグラフで示す。
(例:ダイクストラ法(またはヒューリスティックなし)の場合との計算時間比較。)

\section{考察}
実験結果から何が言えるかを考察する。
(例:設計したヒューリスティック関数はどの程度有効であったか。計算時間がかかったケースがあれば、その原因は何か。)

%%%%%%%%%%%%%%%%%%%%%
% 6章
%%%%%%%%%%%%%%%%%%%%%
\chapter{結論} \label{chapter:conclusion}

本章では、研究全体の成果を総括し、今後の展望を述べる。

\section{本研究の総括}
本研究では、JR分割乗車券の最適解を探索するプログラムをA*アルゴリズムを用いて開発した。設計したヒューリスティック関数を用いることで、現実的な時間内での最適解の算出が可能であることを示した。実験により、特定のケースにおいて、従来の手法(または手計算)よりも効率的に最適解を導出できることを確認した。

\section{将来の展望と今後の課題}
\begin{itemize}
    \item 展望:新幹線や特急料金を含めた最適化、Webサービスとしての一般公開、モバイルアプリ化など。
    \item 課題:運賃改定への自動追従システムの構築、計算速度の更なる向上、未対応の特殊ルール(例:往復割引、ジパング倶楽部など)への対応。
\end{itemize}

%%%%%%%%%%%%%%%%%%%%%
% 謝辞
%%%%%%%%%%%%%%%%%%%%%
\syaji
\par
指導教員、研究室の仲間、その他お世話になった方々への感謝を記述する。

%%%%%%%%%%%%%%%%%%%%%
% 参考文献
%%%%%%%%%%%%%%%%%%%%%
\begin{thebibliography}{99}
  \addcontentsline{toc}{chapter}{参考文献}
  
  \bibitem{bunken1}
  [文献1]
  
  \bibitem{bunken2}
  [文献2]
  
  \bibitem{web1}
  [Webサイト名, URL, (閲覧日)]

\end{thebibliography}

%%%%%%%%%%%%%%%%%%%%%
% 付録
%%%%%%%%%%%%%%%%%%%%%
\appendix
\chapter{付録}

\section{複雑な運賃計算規則の詳細リスト}
本文中では概略に留めた、運賃計算の例外ルール(例:特定区間の詳細)などをまとめる。

\section{ヒューリスティック関数の許容性に関する数学的証明}
(もしあれば)

\section{実装したプログラムの主要なコード片}
(もしあれば)

\end{document}