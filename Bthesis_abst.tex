\RequirePackage{plautopatch}
\documentclass[uplatex,dvipdfmx,10pt,a4paper,notitlepage,oneside,twocolumn]{abst_jsarticle}
% notitlepage : \titlepage は独立しない
% oneside : 奇数・偶数ページは同じデザイン
% twocolumn : 2段組
\usepackage[dvipdfmx]{graphicx}
\usepackage[dvipdfmx,x11names]{xcolor}
\usepackage{amsmath,amssymb}
\usepackage{booktabs}
\usepackage{url}
\usepackage[hang, small, bf, labelsep=quad]{caption}
\usepackage[hmargin=2truecm, textheight= 78zw]{geometry}
\columnsep=\dimexpr \textwidth - 50zw \relax
\title{
\textbf{\textgt{JR最安分割乗車券探索システムの開発}}
}
\author{\begin{center}
{\large \textbf{\textgt{22D8102015D 東條 涼平}}}\\
{\large \textbf{\textgt{中央大学理工学部情報工学科 アルゴリズム理論基礎研究室}}}\\
{\large \textbf{\textgt{2026年3月}}}
\end{center}}
\date{}
\pagestyle{empty}
\begin{document}
\maketitle
%%%%%%%%%%%%%%%%%%%%%%%%%%%%%%
\begin{abstract}
JR線を利用する際,乗車区間を途中で区切って複数枚の乗車券を購入する「分割乗車券」を利用する等を行うことで,通しで購入するよりも運賃が安くなる場合がある.
本研究の目的は,利用者が経路を指定することなく,発駅と着駅の入力のみから分割乗車券を含む最安経路を自動的に導出するシステムを構築することである.
今回構築したシステムにより,既存サービスでは見逃されていた迂回路による最安解を特定できることを確認した一方,長距離区間における計算量のリスクを定量的に明らかにした.
\end{abstract}

\vspace{1zw} \noindent
\textbf{\textgt{キーワード: }}分割乗車券,経路探索,JR運賃計算,グラフ理論.
\section{序論} \label{sec:intro}
\subsection{研究の背景と目的}
JRの旅客運賃は営業キロに基づく階段状の体系をとるため,乗車区間を途中で区切り,複数枚の乗車券を組み合わせる「分割乗車券」を利用することで,運賃を安価に抑えられる場合がある.
また,JRの運賃制度の特例により,経路を外方へ延伸することで運賃が下がる「逆転現象」も存在する.
このように運賃決定の要因は極めて複雑であり,さらに任意の経路上には駅数に対して指数関数的な分割パターンが存在する.
そのため,利用者が迂回経路や逆転現象を含めた全ての選択肢から手動で最安値を見つけることは事実上不可能である.
既存の検索システム\cite{oba}においても,利用者が経路を指定する必要があるなどの課題があった.
本研究では,発駅と着駅のみを入力とし,これらの複雑な運賃規則を厳密に考慮した上で,経路指定を必要とせずに真の最安値を自動探索するシステムの開発を目的とする.

\subsection{最安片道乗車券と最安片道運賃}
通常,片道乗車券は発着駅間の実乗車経路に基づいて発券されるが,前節で示した通り,経路を外方へ延伸した片道乗車券の方が安価となる逆転現象が存在する.
本研究における「最安片道乗車券」とは,入力された経路を含み,かつ旅客営業規則\cite{ryokaku}に則って乗車することができる全ての片道乗車券の中で,その運賃が最も安価となるものを指す.
また,その運賃のことを「最安片道運賃」と定義する.
つまり,最安片道乗車券を用いれば,乗車区間の延伸による運賃の減少という逆転現象が起こることはない.

\subsection{分割乗車券}
発駅から着駅までの経路が与えられたとき,乗車区間の途中駅で片道乗車券を区切って購入し,複数枚の片道乗車券を組み合わせてその経路を乗車することのできる片道乗車券の集合を「分割乗車券」と定義する.
分割を行わない1枚の乗車券も,分割乗車券として考える.
発駅と着駅の間に $n$ 個の途中駅が存在する経路を考えると,各途中駅において「分割する」か「分割しない」かの2通りの選択肢が存在するため,その経路における分割乗車券の総数は $2^n$ 通りとなる.

\subsection{最安分割乗車券と最安分割運賃}
JRの運賃は距離区分ごとに変動する階段状の体系であるため,購入したほうが合計運賃が安価になる場合がある.
ある発着駅間を結ぶ特定の経路において,考えうる $2^n$ 通りの分割パターンのうち,構成する各片道乗車券の運賃の総和が最小となる組み合わせを,当該経路における「最安分割乗車券」と定義する.
ここで,運賃計算の特性上,分割駅の組み合わせが異なっても合計運賃が同額となる場合があるため,「最安分割乗車券」は一つの経路に対して複数存在し得る.
また,その際の最小となる運賃の総和を,当該経路の「最安分割運賃」と定義する.

\section{提案手法} \label{sec:algorithm}
本研究の目的は,JRの複雑な運賃規則を完全に網羅した上で,移動コストを最小化する最安分割運賃の厳密解を導出することである.
そこで本研究では,解の正当性を保証するため,Dijkstra Algorithm\cite{dijkstra}とYen's Algorithm\cite{yen}を組み合わせ,以下の手順で厳密解を導出する手法を提案する.
\\
\begin{enumerate}
    \item \textbf{基準経路の探索と距離閾値の算定}\\
    まず,Dijkstra Algorithm\cite{dijkstra}を用いて発駅から着駅までの運賃計算キロが最短となる基準経路を探索する.
    この基準経路の距離と通し運賃を取得し,これらを基に探索を打ち切るための距離閾値を算出する.
    閾値の算定にあたっては,基準経路の距離以下の範囲で成立しうる最安キロ単価を用い,さらに特定都区市内制度や経路特定区間による距離の乖離を吸収するための補正項を加算する.
    この補正項により,物理的な距離は長いが特例の適用によって運賃が安価になる経路を探索範囲に確実に含めることが可能となる.

    \item \textbf{候補経路の網羅的列挙}\\
    次に,Yen's Algorithm\cite{yen}を用いて,第1最短経路から順に候補経路を列挙する.
    この列挙プロセスは,得られた経路の運賃計算キロが前行程で設定した距離閾値を超えるまで継続する.
    これにより,理論上「最安分割運賃」になり得る全ての候補経路を漏れなく抽出する.
    物理的な距離が閾値を超える経路は,どのような分割パターンを用いても基準経路の運賃を下回ることがないため,この段階で探索対象から除外される.

    \item \textbf{最終的な解の選定}\\
    全ての候補経路から算出された最安分割運賃を比較し,その中で最小となる経路と分割パターンの組み合わせを最終的な厳密解として出力する.
    もし複数の経路で同額の最安運賃が得られた場合には,その全てを解として提示する.
\end{enumerate}

\section{評価実験} \label{sec:evaluation}
開発したシステムの有用性を「正当性」「有用性」「性能」の3つの観点から評価した.

\subsection{正当性および有用性評価}
無作為に抽出した100区間において,公式の運賃検索サービス「えきねっと\cite{ekinet}」と照合し,複雑な特例や逆転現象を含む全ての区間で運賃が一致することを確認した.
また,既存の分割計算サービス\cite{oba}との比較を行った.
東京駅から仙台駅までにおいて,既存サービスでは最短経路を指定した場合の最安分割運賃は5,830円が出力されるが,本システムは赤羽駅から大宮駅までを埼京線を利用する迂回経路を自動的に発見し,その経路の最安分割運賃として5,800円という厳密な最安解を導出した.

分割購入によって運賃が低減する要因は多岐にわたるが,本アルゴリズムの特筆すべき成果として,複数の制度が複雑に絡み合うケースを最適化できた点が挙げられる.
具体的な事例として,中央本線の新宿駅から勝沼ぶどう郷駅までの経路を考える.
通常の片道乗車券である「東京山手線内→勝沼ぶどう郷」の1,980円に対して,高尾駅での分割後の合計運賃は1,570円となった.
これにより,410円(約21\%)の大幅な運賃削減が可能であることを確認した.
本区間では一定距離を超えた場合に中心駅である東京駅からの運賃計算となる特定都区市内制度の適用回避と,競合の私鉄との対抗のために設定されている特定運賃の適用という2つの要因が複合的に作用する.
本事例は,独立した複数の運賃ルールを横断的に探索し,最適解を導出できる本手法の有効性を実証するものである.

\subsection{性能評価}
探索距離に応じた計算時間の推移を測定した.
実験の結果,探索距離が300kmを超える東京→名取において,計算時間が30分を超過するケースが確認された.
これは,グラフの次数に比例して探索空間が爆発的に拡大するためである.

\section{結論} \label{sec:conclusion}
本研究では,経路指定を必要とせず,複雑なJR運賃規則を網羅した最安分割乗車券の自動探索システムを開発した.
評価実験により,既存サービスでは見つけられない迂回経路を含む真の最安分割運賃を導出可能であることが実証された.
今後の課題として,大都市圏における計算時間の増大を抑制するため,運賃逆転が起こり得ない経路を早期に除外する厳格な枝刈り手法の導入が挙げられる.

% 参考文献
\begin{thebibliography}{99}
    \bibitem{dijkstra}
    E.W. Dijkstra, A note on two problems in connexion with graphs, \textit{Numerische Mathematik}, vol.~1, pp.~269--271, 1959.
    \bibitem{ekinet}
    JR東日本,えきねっと,入手先〈\url{https://www.eki-net.com/personal/top/index}〉(参照2025-12-25).
    \bibitem{ryokaku}
    JR東日本,旅客営業規則,入手先〈\url{https://www.jreast.co.jp/ryokaku/}〉(参照2025-12-25).
    \bibitem{oba}
    oba,乗車券分割プログラム,入手先〈\url{https://bunkatsu.jp/}〉(参照2025-12-25).
    \bibitem{yen}
    J. Y. Yen, Finding the K Shortest Loopless Paths in a Network, \textit{Management Science}, vol.~17, no.~11, pp.~712--716, 1971.
\end{thebibliography}
\end{document}